%%%%%%%%%%%%%%%%%%%%%%%%%%%%%%%%%%%%%%%%%%%%%%%%%%%%%%%%%%%%
%									     %
%  R. Van de Water's Research Statement			     %
%  Based on M. Wingate's and David Rainwater's CVs 	     %
%  version of 14 Oct 2004					     %
%									     %
%%%%%%%%%%%%%%%%%%%%%%%%%%%%%%%%%%%%%%%%%%%%%%%%%%%%%%%%%%%%


\documentclass[12pt]{article}

\usepackage{latexsym,setspace}

\setlength{\oddsidemargin}{0in}
\setlength{\topmargin}{-0.5in}
\setlength{\textwidth}{6.5in}
\setlength{\textheight}{9in}
\setlength{\parskip}{0in}
\newcommand{\hr}{\centerline{\hskip 30mm\hrulefill\hskip 30mm}}
\newcommand{\dzero}{D\O}

\def\CO{{\cal O}}

\begin{document}

\begin{center}
{\LARGE \textsc{Thomas Gadfort}}
\smallskip

\emph{Research Statement}
\end{center}

\vspace{-11pt}
\hr

\bigskip

\section{Overview}

I am broadly interested in fundamental particle interactions at the electroweak symmetry breaking energy scale and beyond. Our current understanding of physics at these small distance scales is encapsulated in the Standard Model (SM) of particle physics. While this model remains a crowning achievement of theoretical physics, there are many reasons to believe that nature has a more fundamental structure. Specifically, the model offers no explanation for the recent observations of neutrino mass, dark matter and dark energy. Nor does it clarify the role of quark and lepton flavor or the disparity in strength between gravity and the three other fundamental forces.

To answer these questions experiments must probe nature at very small distance scales. The high-energy physics community is pursuing the search for physics beyond the Standard Model with two complementary approaches.  The first is via direct production of new particles at high-$p_{\rm{T}}$ colliders such as the Large Hadron Collider (LHC).  The second is via precise measurements of particle interactions using intense low-energy accelerators such as the NuMi neutrino beam at Fermilab or the PEP-II $e^+e^-$ collider at SLAC and looking for deviations from the Standard Model predictions. My research to-date has focused on the former approach to measure the rate of electroweak top-quark production and search for resonant structure in diboson particle production. As an associate scientist at Fermilab, I will continue my research following the latter approach through a precise measurement of the muon magnetic moment, a search for lepton flavor violation in muon-to-electron conversions, or a determination of the neutrino mass hierarchy by comparing the oscillation rates of muon neutrinos to muon antineutrinos. The following is a brief description of my experience in high-energy collider experiments and my research interests as related to the physics program at Fermilab.

\section{High-energy Collider Research Experience}

\subsection{Physics Analysis}
My research career in high-energy particle physics began as a graduate student on the Run II \dzero~experiment, and continued as a postdoc on both the \dzero~and ATLAS experiments. For my Ph. D. thesis topic, I chose to search for electroweak top-quark production, commonly called ``single top", using a powerful analysis technique known as the matrix element method. This technique, which uses QFT matrix elements to categorize signal and background events, was pioneered at \dzero~during Run I and, at the time, lead to the most precise measurement of the top-quark mass. Using $1$~fb$^{-1}$~of Run II data I was the principle analyzer that established the first $3\sigma$~evidence for this process. As a postdoc, I  managed an effort to search for resonant structure in events with two $W$~bosons or one $W$~and one $Z$~boson. I demonstrated that a quantity known as jet mass is crucial for reconstructing very high-mass signal events and published the first result on this topic at the Tevatron. As a member of the ATLAS collaboration for the past three years I expanded the diboson search to cover more decay channels thereby increasing sensitivity to a new physics signal. In addition to the diboson effort, I supervised a Stony Brook University graduate student on a search for leptoquarks, which are hypothetical particles that carries both baryon and lepton flavor. Our work improved upon the already stringent limits set by the Tevatron experiments. I recently updated this search using $1$~fb$^{-1}$~of integrated luminosity collected by the ATLAS detector. 

\subsection{Hardware Experience}
I have been involved in several hardware projects on both the \dzero~and ATLAS experiments. As a graduate student on the \dzero~experiment I maintained and improved reliability of the data acquisition (DAQ) system, which is based on single-board computers (SBCs) connected to the sub-detector readout electronics. During my tenure with the DAQ group I was responsible for the distribution of the online software which determines if an event will be stored on tape. I also upgraded the online monitoring software to improve shifter awareness of the DAQ status and trigger accept rates. As a member of the ATLAS collaboration I built a multiplexing electronic circuit with nano-Ampere sensitivity to characterize transistors formed on a Silicon-Germanium (SiGe) substrate. These transistors were measured before and after exposure to high levels of neutron and gamma radiation thus simulating the operational environment expected at the upgraded LHC with instantaneous luminosities above \mbox{$10^{35}$~cm$^{-2}$ s$^{-1}$}. During this time I also constructed a simplified version of the next generation ATLAS calorimeter front-end electronics capable of recording a $40$~MHz analog signal and recording the converted digital format in realtime. 

\subsection{Leadership Positions}
In addition to my data analysis and hardware efforts, I have served in two high-profile leadership positions within the \dzero~and ATLAS collaborations. From 2007 to 2009 I convened the $b$-jet identification group at \dzero, which is charged with developing algorithms to distinguish jets originating from a bottom ($b$) quark from the more common up-down-strange ($uds$) and gluon ($g$) jets. Under my leadership our group developed a powerful algorithm to select $b$-jets based on neural networks, which greatly improved the low-mass Higgs search sensitivity. During this time I supervised the work of several graduate students and fellow  postdocs and presented these efforts to the \dzero~management at weekly meetings. Earlier this year I convened the first group to search for diboson resonances using ATLAS data. For this effort I coordinated and presided over discussions between the group and the ATLAS collaboration management. I am also the editor of two publications for this group based on a search for technicolor in three-lepton events and a model-independent search for $ZZ$~resonances using four-lepton events. This involved both writing sections of the paper and guiding it through the internal ATLAS review process. I believe these leadership experiences have prepared me well to work in, and eventually play a leading role in, future collaborations.

\section{Future Research Plans in Intensity Physics}
Precise measurements of known particle properties such as the muon magnetic moment or searches for rare decays such as $K \rightarrow \pi\nu\bar{\nu}$ or $\mu\rightarrow e$~are an excellent probes for beyond the SM (BSM) physics and are complementary to direct searches for new particle production at the energy frontier. Once new physics is observed at the LHC, we will measure the low-lying particle spectrum, but we may still be unable to determine the underlying theory because there are so many candidate new physics models with similar experimental signatures:  this is known as the ``LHC inverse problem". For example, if a top-antitop quark resonance is observed at the LHC, we will not know if the new particle arrises from extra dimensions, grand unification, or new strong dynamics, because all of these models predict such resonances with comparable production rates. A precise measurement of the muon magnetic moment is critical because it can help distinguish the underlying physics since some BSM theories generate significant contributions to its value ({\it{e.g.}}~supersymmetry (SUSY) or warped extra dimensions) while others do not ({\it{e.g.}}~universal extra dimensions). Another reason experiments at the intensity frontier may be critical to high-energy physics research is if the ``nightmare scenario" occurs in which the LHC only discovers a low-mass Higgs boson. If this is the case then the only way to probe the physics that stabilizes the Higgs mass term is through virtual quantum corrections to known particle properties or decay rates. For example, SUSY may exist, but its couplings and mass states may be arranged such that the detection at the LHC is impossible (e.g. little missing transverse energy due to small mass splittings). SUSY can, however, enhance the decay rate of $K \rightarrow \pi\nu\bar{\nu}$~or $\mu \rightarrow e$~conversion thus making a high-statistics measurement of these processes crucial to understanding nature above the TeV-scale.

Over the next several years Fermilab will be carrying out an exciting program of high-energy physics research. As an associate scientist at Fermilab I will continue my high-energy physics career carrying out experiments at the intensity frontier. I am especially interested in three upcoming experiments at Fermilab: muon $g-2$, Mu2e, and NOVA/LBNE, each of which I discuss below. 

\subsection{g-2}
The new muon $g-2$~experiment proposes to increase the statistics by a factor of twenty over the recent BNL muon $g-2$~experiment, which to-date reports a $>$$3\sigma$~discrepancy with the SM prediction of the muon magnetic moment $(g-2)_{\mu}$. Coupled with improved theoretical understanding of hadronic contributions using lattice gauge theory, this experiment could observe a $>$$7\sigma$ deviation from the Standard Model assuming fixed theoretical and experimental central values. If SUSY is the fundamental theory of physics at the TeV-scale, the $(g-2)_\mu$~result will significantly constrain the allowed values of the 105 model parameters, which the LHC has little hope to accomplish. In additional, $(g-2)_\mu$~can resolve the sign of the $\mu$-term in the SUSY Lagrangian. Another important aspect of the measurement is that $(g-2)_\mu$~is sensitive to the supersymmetric parter of the muon ($\tilde{\mu}$), a particle that is difficult to measure at the LHC due to its hadronic initial state. The $g-2$~experiment is also sensitive to a non-zero muon electric dipole moment (EDM). Such an observation will constitute the first measurement of time-symmetry ($T$) violation in the lepton sector and serve as an additional source of charge-parity ($CP$) violation assuming that the combined $CPT$~symmetry is still valid. Large muon EDMs are predicted by many BSM models including the proposed see-saw mechanism responsible for neutrino mass generation.

\subsection{Mu2e}
Another experiment that interests me is the search for charged lepton flavor violation (CLFV) through muon-to-electron ($\mu\rightarrow e$)~conversions in the presence of an Aluminum nucleus. The Mu2e experiment expects to measure the rate of $\mu^{-}+\rm{Al}\rightarrow e^{-}+\rm{Al}$ to one part in $\sim2\times10^{16}$, representing a four orders of magnitude improvement over the current limit set by the \mbox{SINDRUM II}~experiment. The SM allows for such conversions via neutrino oscillations although the rate is below one part in $10^{54}$~due the extremely small neutrino mass. SUSY models as well as models with leptoquarks and new $Z'$~bosons predict CLFV with rates well above the expected sensitivity of this experiment. An upgrade of the detector coupled with an increased proton luminosity expected with the Project X  proton beam can increase the sensitivity for ($\mu\rightarrow e$)~conversions to one part in $10^{18}$. Because the conversation rate depends on the target nuclei, a different target nuclei can be used ({\it{e.g.}}~gold) in the upgrade to further constrain the underlying physics.

\subsection{NOVA}
The NOVA experiment is also extremely interesting given the recent results of muon neutrino ($\nu_{\mu}$) oscillations into electron neutrinos ($\nu_{e}$), and thereby of the neutrino mixing angle $\theta_{13}$, from the T2K, Double Chooz, and Minos experiments. Using the off-axis NuMi beamline at Fermilab this experiment will make a very precise measurement of $\sin^{2}(\theta_{13})$~using two detectors, one located onsite at Fermilab and one located over 700 kilometers away in Northern Minnesota. NOVA can also establish the ordering of the neutrino mass states through a precise measurement of the $\nu_{\mu}\rightarrow\nu_{e}$~oscillations rate. In addition to the mixing terms, NOVA is sensitive to a non-zero $CP$-violating phase because the NuMi beamline can produce both muon neutrinos and muon antineutrinos ($\bar{\nu}_{\mu}$). The long baseline neutrino experiment (LBNE) is the natural next-generation experiment for neutrino measurements and searches for proton decay. Coupled with Project X this large underground detector will make definitive measurements of neutrino mixing parameters, continue the search for supernova neutrinos, and dramatically increase the sensitivity for baryon violation through proton decay.

\clearpage
\section{Summary}
As an associate scientist I will play a key role in high-energy intensity physics experiments at Fermilab. All of these experiments are exciting because they have the potential to observe new physics within the next few years. Further, they will play an important role in interpreting observations of new particles directly produced at the LHC and in identifying the true theory of physics beyond the Standard Model. These experiments also serve as a stepping stone for the long-term future of Fermilab in the post-LHC era, which may include a new high-energy collider such as the international $e^{+}e^{-}$~linear collier or a higher-energy muon collider. I believe my experience in high-energy collider research will translate very well to these experiments since they too require large data acquisition systems, sophisticated techniques to analyze large data volumes, and demand positive interpersonal skills to function within a large collaboration. Whatever I choose to work on, however, will further my central goal to observe and understand the fundamental particles and their interactions at the electroweak scale and beyond.

\end{document}