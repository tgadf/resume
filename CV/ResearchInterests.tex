%%%%%%%%%%%%%%%%%%%%%%%%%%%%%%%%%%%%%%%%%%%%%%%%%%%%%%%%%%%%
%									     %
%  R. Van de Water's Research Statement			     %
%  Based on M. Wingate's and David Rainwater's CVs 	     %
%  version of 14 Oct 2004					     %
%									     %
%%%%%%%%%%%%%%%%%%%%%%%%%%%%%%%%%%%%%%%%%%%%%%%%%%%%%%%%%%%%


\documentclass[12pt]{article}

\usepackage{latexsym,setspace}

\setlength{\oddsidemargin}{0in}
\setlength{\topmargin}{-0.5in}
\setlength{\textwidth}{6.5in}
\setlength{\textheight}{9in}
\setlength{\parskip}{0in}
\newcommand{\hr}{\centerline{\hskip 30mm\hrulefill\hskip 30mm}}
\newcommand{\dzero}{D\O}

\def\CO{{\cal O}}

\begin{document}

\begin{center}
{\LARGE \textsc{Thomas Gadfort}}
\smallskip

\emph{Research Interests}
\end{center}

\vspace{-11pt}
\hr

\bigskip
 
I am broadly interested in fundamental particle interactions at the electroweak symmetry breaking energy scale. The origin of electroweak symmetry breaking (EWSB) is unknown, however, the standard model of particle physics provides one possibility through what is commonly called the Higgs mechanism. A massive scalar particle commonly called the Higgs boson is predicted by this mechanism, however its mass remains unconstrained within the standard model and it has yet to be observed. Even if the Higgs boson is discovered there are several compelling reasons to believe in additional physics beyond the standard model.

This scenario 
Specifically, 


Our knowledge of physics below this scale is encapsulated in the standard model, which describes the electromagnetic, weak, and strong nuclear forces as fundamental interactions between Fermionic matter particles and Bosonic force carriers. The strong force is mediated by eight massless particles known as gluons  and their interactions are described by the $\rm{SU}(3)$ symmetry group. In the standard model, the electromagnetic and weak forces are unified at energy scales much higher than those of our everyday world and they are described by a $\rm{SU}_{\rm{L}}(2) \times \rm{U}_{\rm{Y}}(1)$ symmetry group with four force carries: two electrically neutral and two electrically charged. At lower energies these forces decouple in a process known as electroweak symmetry breaking (EWSB) resulting in one force carrier of the electromagnetic interaction, the photon, and three force carriers of the weak interaction, the $W^{\pm}$, and $Z$~boson. The unification of the electromagnetic and weak forces and the subsequent observation of the $W$~and $Z$~bosons at CERN remains one of the greatest achievements in theoretical and experimental physics during the second half of the twentieth century.

The origin of electroweak symmetry breaking is unknown. The standard model provides one possibility through the introduction of a new field, whose expectation value allows for massive $W$~and $Z$~bosons and a massless photon. This scenario of EWSB is commonly called the Higgs mechanism and requires one additional spin-0 particle commonly called the Higgs boson. While the Higgs mechanism provides an attractive solution to EWSB, it, unfortunately, does not give any guidance for the value of the Higgs boson mass. Direct searches for Higgs production at LEP suggest that its mass must be greater than $114.4$~GeV and a myriad of precision measurements of electroweak standard model parameters suggest that its mass must also be less than 159~GeV. Recently, direct searches for the Higgs boson at the Tevatron exclude masses in the range between 158 and 176~GeV.

It is widely believed that the Higgs boson, if it exists, will be discovered in the coming years, but there are many reasons to believe that the standard model Higgs is unlikely to be the only new physics realized in nature above the EWSB scale. The most compelling reason why this must be true is observed when computing quantum corrections to the Higgs boson mass. Because the Higgs field couples more strongly to higher mass objects, corrections to its mass are naturally set by the nearest energy scale above EWSB. In the absence of other physics, the nearest scale is the Plank scale, which is nineteen orders of magnitude larger than the EWSB scale, suggesting that some other new energy scale must exist near the EWSB scale to keep the Higgs mass within the experimentally allowed region. Another reason why the Higgs mechanism is unlikely to be the sole new physics beyond the standard model is that it provides no explanation for the origin of Fermionic mass. In the standard model fermions acquire mass through a Yukawa interaction with the Higgs field, but the scale of that interaction is a unique value for each fermion, thus there is no explanation for the tremendous difference between the lightest and heaviest fermions. ** might need more **

As a Wilson Fellow, I plan to continue searching for physics beyond the standard model. During the past few years I have focused my attention that must exist for EWSB at or below the TeV-energy scale. 



 either at the Tevatron or the LHC in the coming years.





 and it's discovery would be a triumph  

destroys the $\rm{SU}_{\rm{L}}(2) \times \rm{U}_{\rm{Y}}(1)$ symmetry
 that acquires an expectation value resulting in 

Higgs or Englert-Brout-Higgs-Guralnik-Hagen-Kibble mechanism.

We have long known however that this symmetry is not realized in nature since the weak force carries, the $W^{\pm}$~and $Z^{0}$~bosons, are not massless thus violating the underlying symmetry group of that interaction. The solution 
Therefore, this symmetry must be broken  


. successful prediction of neutral current interactions 


are mediated by particles known as gauge Bosons. In this 

are the result of underlying symmetries in nature. These symmetries 

In this model, the strong nuclear force is described by a non-Abelian gauge theory based on the unbroken symmetry group $SU(3)$~whereas the electromagnetic and weak nuclear force are the result of a broken symmetry group $SU_{L}(2)\times U_{Y}(1)$.

on the basis of symmetry 
There are many reasons to believe that the standard mo

To study this, my research thus far has focused on collider-based searches for physics predicted by the standard model that remains unobserved and direct searches for new fundamental particles, whose observation would indicate the need for physics beyond the standard model. 

My research interests fall broadly into fundamental particle interactions at energies at or above the 

at energy scales above the tera-electro
experimental high energy physics. Many of the 

\end{document}