%%%%%%%%%%%%%%%%%%%%%%%%%%%%%%%%%%%%%%%%%%%%%%%%%%%%%%%%%%%%
%									     %
%  R. Van de Water's Research Statement			     %
%  Based on M. Wingate's and David Rainwater's CVs 	     %
%  version of 14 Oct 2004					     %
%									     %
%%%%%%%%%%%%%%%%%%%%%%%%%%%%%%%%%%%%%%%%%%%%%%%%%%%%%%%%%%%%


\documentclass[12pt]{article}

\usepackage{latexsym,setspace}

\setlength{\oddsidemargin}{0in}
\setlength{\topmargin}{-0.5in}
\setlength{\textwidth}{6.5in}
\setlength{\textheight}{9in}
\setlength{\parskip}{0in}
\newcommand{\hr}{\centerline{\hskip 30mm\hrulefill\hskip 30mm}}
\newcommand{\dzero}{D\O}

\def\CO{{\cal O}}

\begin{document}

\begin{center}
{\LARGE \textsc{Thomas Gadfort}}
\smallskip

\emph{Research Statement}
\end{center}

\vspace{-11pt}
\hr

\bigskip
 
I am broadly interested in fundamental particle interactions at the electroweak symmetry breaking energy scale and beyond. Although the true origin of electroweak symmetry breaking (EWSB) is unknown, the standard model of particle physics provides one possibility through what is commonly called the Higgs mechanism. The standard model predicts the existence of a single massive particle, but its mass is unconstrained by the theory and the Higgs boson has yet to be observed.

There are many reasons to suggest that the standard model Higgs is not the only new physics realized in nature. The most compelling evidence for physics beyond the standard model Higgs is seen from quantum corrections to its mass. Because the Higgs field couples more strongly to more massive objects, corrections to its mass are naturally set by the nearest energy scale above the electroweak scale. In the absence of other physics, the nearest energy scale is the Plank scale, which is nineteen orders of magnitude larger than the electroweak scale, suggesting that some other new energy scale must exist near the electroweak scale to keep the Higgs mass within the experimentally allowed region.

Over the past few years, I have searched for several candidates for physics beyond the standard model and I would like to continue these efforts as a Wilson Fellow. Here I discuss two extensions to the standard model that I find particularly interesting not only from the theoretical justification, but also from the challenging experimental signatures they produce.

One particularly attractive solution that keeps the Higgs mass light came ten years ago when Randall and Sundrum postulated a universe with three large spatial dimensions and one additional highly warped dimension. This new dimension helps keep the Higgs boson light because its mass as observed in three dimensions is actually an exponentially reduced fraction of the Plank mass, where the exponential factor depends on the geometry of the warped dimension. More modern versions of this model provide a natural explanation for the disparity between the fermion masses by postulating that the heavier fermions are actually particles that are more localized near our three dimensional world therefore receive less of an exponential decrease in their observed mass. This scenario also naturally explains why the force of gravity is orders of magnitude weaker when compared to the other three known forces. The force of gravity depends on the geometry of space-time, so it is allowed to propagate into this warped dimension. Because gravity ``feels" this new space it appears weak in our three dimensional world.

During my tenure as a postdoc at Columbia University and then as a Goldhaber Distinguished Fellow at Brookhaven National Laboratory, I led an effort to search for this extra dimension by observing a spin-2 particle known as the Randall-Sundrum (RS) graviton at the Tevatron using Run II data collected with the \dzero~detector. Specifically, I searched for the RS graviton when it decays into a $WW$~boson pair. This final state is experimentally attractive because it leads to highly collimated jets when either of the $W$~bosons decays hadronically. When the $W$~boson transverse momentum becomes much greater than its rest mass we observe that the $W$~boson is more likely to be found as a single jet instead of two isolated jets. Using this unique feature of hadronic $W$~boson decays, I extended the current Tevatron limit on RS graviton production into the $WW$~final state by nearly 100~GeV and helped pioneer the use of the jet mass at the Tevatron as a useful quantity when searching for physics beyond the standard model. 

The warped geometry model also provides a myriad of other predictions at the Large Hadron Collider (LHC) and as a Wilson Fellow this will be my primary area of study. At the LHC, the search for hadronic decays of the $W$~and $Z$~bosons as well as the equally interesting hadronic decays of the top quark become more powerful search channels due to the increased collision energy and the fine granularity of both the ATLAS and CMS calorimeters. One ``smoking gun" signature of warped extra dimensions are Kaluza-Klein (KK) excitations of the standard model particles. These excitations are the result of the finite extra dimension length in analogy to the energy levels of a particle in an infinite well studied in undergraduate quantum mechanics. In particular, the KK excitations of the standard model gluon are predicted to couple more strongly to particles localized closer to our three dimensional world, such as the top quark, as compared to other colored objects. This search is experimentally challenging because the strongly interacting KK gluon has a decay width much larger than the detector resolution. Extracting this signal out of the multijet background requires new jet clustering algorithms such as the Cambridge-Aachen method in tandem with new methods of ``pruning" away soft QCD radiation to improve the overall jet mass resolution. An additional prediction of warped extra dimension models is the existence of a massive neutral resonance that is generically called the $Z'$. In these models, the $Z'$~also couples more strongly to heavier particles in a similar manner as to the KK gluon. If this scenario is true then this particle is most easily seen as a resonance in $e^{+}e^{-}$~and $\mu^{+}\mu^{-}$~final states. Because the muon is two orders of magnitude heavier than the electron we can expect a larger decay width to $\mu^{+}\mu^{-}$ compared to $e^{+}e^{-}$, a scenario not expected in other models containing $Z'$-like particles such as left-right symmetric models or grand unified theories. Observing KK excitations of the gluon and lepton flavor non-universality in $Z'$~decays will be two of my principle goals as a Wilson Fellow.

In addition to searches for warped extra dimensions, I recently began searching for first, second, and third generation leptoquarks. Leptoquarks are bosonic particles that mediate interactions between leptons and quarks and appear in many grand unified theories. My current interest lies with single production of leptoquarks in association with one lepton, either a charged lepton or a neutrino depending on the colliding quark flavor. A unique feature of single leptoquark production at the LHC is the large excess of positively charged leptons compared to negatively charged leptons due to the proton-proton initial state. Because these events result from a gluon-quark collision, they tend to be highly boosted along the direction of the incoming quark. This leads to an experimentally challenging scenario because lepton reconstruction is low in the forward region of the detector. I am currently coordinating an effort within the ATLAS collaboration to extend the Tevatron sensitivity for leptoquark production and I plan to continue this search if selected for a Wilson Fellowship.

Another interest of mine is the implementation of advanced multivariate analysis techniques to search for physics beyond the standard model. I have expert knowledge of many of these methods, such as the matrix element method, and they have proven invaluable at the Tevatron and will also be important at the LHC. A program I would like to implement as a Wilson Fellow at Fermilab is to collaborate with the computing community to explore new ways to improve the performance of these methods. For example, a group of lattice gauge theorists recently employed graphics processing units (GPUs) to dramatically decrease the computation time required to compute the baryon mass spectrum from first principles. I would like to investigate if this or similar techniques can be successfully applied within the high energy physics community.

As a Wilson Fellow I would also like to play a role in future high-energy physics experiments at Fermilab.  For example, my years of experience both from the Tevatron and the LHC will be useful in new physics searches at the energy frontier with the proposed muon collider or at the intensity frontier with the proposed kaon and neutrino experiments.  Whatever I choose to work on, however, will further my central goal to observe and understand the fundamental particles and their interactions at the electroweak scale and beyond.

\end{document}