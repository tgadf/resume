%%%%%%%%%%%%%%%%%%%%%%%%%%%%%%%%%%%%%%%%%%%%%%%%%%%%%%%%%%%%
%
%  Thomas Gadfort's CV
%  Based on R. Van de Water's CV
%  Based on M. Wingate's and David Rainwater's CVs 
%  version of 29 May 2006
%
%%%%%%%%%%%%%%%%%%%%%%%%%%%%%%%%%%%%%%%%%%%%%%%%%%%%%%%%%%%%


\documentclass[12pt]{article}

\usepackage{latexsym,setspace}
\usepackage{enumitem}
\usepackage{color}

%\setstretch{0.5}

\setlength{\oddsidemargin}{0in}
\setlength{\topmargin}{-.5in}
\setlength{\textwidth}{6.5in}
\setlength{\textheight}{9in}
\setlength{\parskip}{0.2in}
%\newcommand{\hr}{\centerline{\hskip 6mm\hrulefill\hskip 6mm}}
\newcommand{\hr}{\centerline{\hskip 30mm\hrulefill\hskip 30mm}}

\newcommand{\dzero}{D\O}
\newcommand{\phd}{Ph.D.}
\newcommand{\gmtwo}{$g-2$}


\begin{document}

\pagestyle{empty}

%\vspace{-1cm}\hspace{12cm}{}\emph{\today}

%%%%%%%%%%%%%%%%%%%%%%%%%%%%%%%%%%%%%%%%%%%%%%%%%%%%%%%%%%%%%%%%%%%%%%%%
\vspace{.5cm}

\begin{center}
{\LARGE \textsc{Thomas Gadfort}}

\vspace{-.3cm}\emph{R$\acute{e}$sum$\acute{e}$}

\hr

%%%%%%%%%%%%%%%%%%%%%%%%%%%%%%%%%%%%%%%%%%%%%%%%%%%%%%%%%%%%%%%%%%%%%%%%
\noindent
{\sc Contact Information}

\begin{tabular}{lcl}
Fermi National Accelerator Laboratory		& \hspace{3.47cm} & Telephone: (206) 351-4510 \\
MS122							& \hspace{2.2cm}   & FAX: (630) 840-2347  \\
Batavia, IL 60510					& \hspace{2.2cm} & E-mail: tgadfort@fnal.gov \\
\end{tabular}
%http://www.linkedin.com/pub/thomas-gadfort/62/355/143 \\

\end{center}


\noindent
{\Large \textbf{Career Objective}}

\noindent I am seeking to build a long-term career in SOMETHING where I can utilize my skills in computer programming, data analysis, communication, and project management.  I have ten years of research experience in the field of experimental particle physics.  I have worked with diverse teams of collaborators from many different countries to analyze large and complex data.  Throughout this time I regularly presented my results and those of colleagues in talks at collaboration meetings and international conferences.  I also documented my research by writing collaboration notes, conference proceedings, and refereed journal publications.  I believe that my background in scientific research will be valuable in the field of SOMETHING.

%%%%%%%%%%%%%%%%%%%%%%%%%%%%%%%%%%%%%%%%%%%%%%%%%%%%%%%%%%%%%%%%%%%%%%%%

%\pagestyle{myheadings}
%\topskip=0.75cm
%\headheight=0.75cm
%\markright{Thomas Gadfort}


\noindent
{\Large \textbf{Professional Experience}}
\begin{itemize}[leftmargin=0.5cm]
\item{{\large{Associate Scientist at Fermi National Accelerator Laboratory (2012~-~Present)}} \\
Member of the new \gmtwo~experiment at Fermilab with 100+ collaborators.
\begin{itemize}
\item{Level 3 Project Manager of the superconducting inflector magnet.}
\item{Established the reliability of a C++ GEANT-based simulation framework.}
\item{Supervisor of an Illinois Math and Science Academy student.}% studying the interaction of magnetism and electric circuits.}
\end{itemize}}

\item{{\large{Goldhaber Fellow at Brookhaven National Laboratory (2009~-~2012)}} \\
Member of the 3000+ member ATLAS~collaboration at the Large Hadron Collider.
\begin{itemize}
\item{Convenor of the ``Diboson Resonance" group charged with searching for new elementary particles.}
\item{Introduced state-of-the-art Monte Carlo-based simulations to improve searches.}
\item{Published three articles in the prestigious American Physical Society journals Physical Review Letters and Physical Review D.}
\item{Supervised two \phd~candidates at Columbia and Stony Brook Universities.}
\end{itemize}}

\item{{\large{Postdoctoral Reseacher at Columbia University (2007~-~2009)}} \\
Member of the \dzero~and ATLAS collaborations.
\begin{itemize}
\item{Constructed a LabView-operated parametric analysis to determine if a silicon-germanium (SiGe) transistor substrate can tolerate high-radiation environments.}
\item{Convenor of the $b$-jet identification group charged with developing tools to detect and calibrate bottom-quark ($b$)~jets.}
\item{Created a highly efficient $b$-jet identification algorithm using the novel boosted decision tree multivariate algorithm.}
\item{Produced a comprehensive internal document on the performance of $b$-jet identification algorithms using Tevatron collider data.}
%\item{Published two articles on searches for new particle production at the Tevatron collider.}
\end{itemize}}

\item{{\large{Research Assistant at University of Washington (2002~-~2007)}} \\
Member of the 700+ member \dzero~collaboration at Fermi National Accelerator Laboratory.
\begin{itemize}
\item{Established the first experimental evidence for a physical process known as ``single top-quark" production.}
\item{Developed optimized C++ software to compute multi-dimensional integrals.} %of Tevatron collider data.}
\item{On-call expert for high-level trigger and ethernet-based data acquisition system.} %for the \dzero~experiment.}
\item{Presented tutorials on the data acquisition system to collaboration members.}
\end{itemize}}

\end{itemize}

%%%%%%%%%%%%%%%%%%%%%%%%%%%%%%%%%%%%%%%%%%%%%%%%%%%%%%%%%%%%%%%%%%%%%%%%



\noindent
{\Large \textbf{Education}}
\begin{itemize}[leftmargin=0.5cm]
\itemsep0em
\item{ {\textbf{Ph.~D.}} in high-energy particle physics from the University of Washington (2007) \\
Title: ``Evidence for Electroweak Top Quark Production in~$p\bar{p}$~Collisions at {\mbox{$\sqrt{s}=1.96$~TeV}}" \\ Eugene Kenneth Miller Award for graduate research}
\item {\textbf{M.~S.}} in physics from the University of Washington (2003)
\item {{\textbf{B.~A.}} in College Scholars (Physics Emphasis) from the University of Tennessee (2001) \\
Douglas V. Roseberry Award for excellence in Physics, Phi Beta Kappa, Sigma Pi Sigma Physics Honors Society, Outstanding First Year Physics Student}
\end{itemize}


%%%%%%%%%%%%%%%%%%%%%%%%%%%%%%%%%%%%%%%%%%%%%%%%%%%%%%%%%%%%%%%%%%%%%%%%
\noindent
{\Large \textbf{Computing Experience}}

\begin{itemize}[leftmargin=0.5cm]
\itemsep0em

\item{
\begin{tabular}{ll}
{\textbf{Office Applications}}		&		MS Word, Excel, PowerPoint; Mac Keynote, Pages, Numbers \\
\end{tabular}}

\item{
\begin{tabular}{ll}
{\textbf{Operating Systems}} 		&		Windows 2000/NT/XP/7, Linux/Unix, Macintosh OS 10.x \\
\end{tabular}}

\item{
\begin{tabular}{ll}
{\textbf{Programming Langauges}}	&		C, C++, FORTRAN, Visual Basic, Java \\
\end{tabular}}

\item{
\begin{tabular}{ll}
{\textbf{Scripting Langauges}}		&		Perl, BASH, and Python \\
\end{tabular}}

\item{
\begin{tabular}{ll}
{\textbf{Web Development}} 		&		HTML, PHP, JavaScript \\
\end{tabular}}

\item{
\begin{tabular}{ll}
{\textbf{Database Applications}} 	&		MySQL, XML, Microsoft Access \\
\end{tabular}}

\item{
\begin{tabular}{ll}
{\textbf{Scientific Applications}} 	&		ROOT, GEANT, GSL, Mathematica, LabView \\
\end{tabular}}

\end{itemize}



\noindent
{\Large \textbf{References}}

\begin{minipage}[t]{.45\textwidth}

{\sc Professor Gordon Watts} \\
University of Washington\\
Department of Physics\\
P.O. Box 351560\\
Seattle, WA 98195-1560\\ 
USA\\[0.1cm]
Telephone: (206) 543-4186\\
FAX:\phantom{iiiiiiiii} (206) 616-9172\\
E-mail: gwatts@phys.washington.edu\\

{\sc Professor Gustaaf Brooijmans} \\
Columbia University\\
Nevis Laboratories \\
136 South Broadway\\
Irvington, NY 10533\\
USA\\[0.1cm]
Telephone:  (914) 591-2804 \\
FAX:\phantom{iiiiiiiii} (914) 591-8120\\
E-mail: gusbroo@nevis.columbia.edu\\

{\sc Marc-Andre Pleier} \\
Brookhaven National Laboratory\\
Physics Department, Building 510A\\
Upton, NY 11973\\
USA\\[0.1cm]
Telephone:  (631) 344-4249 \\
FAX:\phantom{iiiiiiiii} (631) 344-5078\\
E-mail: mpleier@bnl.gov\\

%\end{minipage}
%\begin{minipage}[t]{.45\textwidth}

%{\sc Professor John Hobbs} \\
%Physics and Astronomy Department \\
%Stony Brook University \\
%Stony Brook, NY 11794-3800 \\
%USA \\[0.1cm]
%Telephone: (631) 632-8107 \\
%FAX:\phantom{iiiiiiiii} (631) 632-8176\\
%E-mail: John.Hobbs@stonybrook.edu \\

%{\sc Dr. Aurelio Juste} \\
%Institut de F�sica d'Altes Energies (IFAE) \\
%Edifici Cn \\
%Universitat Aut�noma de Barcelona \\
%E-08193 Bellaterra (Barcelona) \\
%Spain \\[0.1cm]
%Telephone:  (+34) 93 581 4249 \\
%FAX:\phantom{iiiiiiiii} (+34) 93 581 1938\\
%E-mail: juste@ifae.es \\

\end{minipage}



\end{document}