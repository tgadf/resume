%%%%%%%%%%%%%%%%%%%%%%%%%%%%%%%%%%%%%%%%%%%%%%%%%%%%%%%%%%%%
%									     %
%  R. Van de Water's Research Statement			     %
%  Based on M. Wingate's and David Rainwater's CVs 	     %
%  version of 14 Oct 2004					     %
%									     %
%%%%%%%%%%%%%%%%%%%%%%%%%%%%%%%%%%%%%%%%%%%%%%%%%%%%%%%%%%%%


\documentclass[12pt]{article}

\usepackage{latexsym,setspace}

\setlength{\oddsidemargin}{0in}
\setlength{\topmargin}{-0.5in}
\setlength{\textwidth}{6.5in}
\setlength{\textheight}{9in}
\setlength{\parskip}{0in}
\newcommand{\hr}{\centerline{\hskip 30mm\hrulefill\hskip 30mm}}
\newcommand{\dzero}{D\O}

\def\CO{{\cal O}}

\begin{document}

\begin{center}
{\LARGE \textsc{Thomas Gadfort}}
\smallskip

\emph{Research Interests}
\end{center}

\vspace{-11pt}
\hr

\bigskip
 
I am broadly interested in fundamental particle interactions at the electroweak symmetry breaking energy scale. The origin of electroweak symmetry breaking (EWSB) is unknown, however, the standard model of particle physics provides one possibility through what is commonly called the Higgs mechanism. A single massive scalar, commonly called the Higgs boson, is predicted by this mechanism, however its mass remains unconstrained within the standard model and it has yet to be observed.

There are many reasons to believe that the standard model Higgs is unlikely to be the only new physics realized in nature. The most compelling evidence for physics beyond the standard model Higgs is seen from quantum corrections to its mass. Because the Higgs field couples more strongly to more massive objects, corrections to its mass are naturally set by the nearest energy scale above the electroweak scale. In the absence of other physics, the nearest energy scale is the Plank scale, which is nineteen orders of magnitude larger than the electroweak scale, suggesting that some other new energy scale must exist near the electroweak scale to keep the Higgs mass within the experimentally allowed region.

One particularly attractive solution to this problem came ten years ago when Randall and Sundrum postulated a universe with three large spatial dimensions and one additional highly warped dimension. This new dimension helps keep the Higgs boson light because its mass as we see it in three dimensions is actually an exponentially reduced fraction of the Plank mass, where the exponential factor depends on the geometry of the warped dimension. More modern versions of this model provide a natural explanation for the disparity between the fermion masses by stating that the heavier fermions are actually particles that are more localized in our three dimensional world thus receiving less of an exponential decrease in the observed mass. This scenario also naturally explains why the force of gravity is orders of magnitude weaker when compared to the other three known forces. Because the force of gravity depends on the geometry of space-time, it is allowed to propagate into this warped dimension. Because gravity ``feels" this new space it appears weak in our three dimensional world.

During my tenure as a postdoc at Columbia University and then as a Distinguished Goldhaber Fellow at Brookhaven National Laboratory, I led an effort to search for this extra dimension by observing a spin-2 particle known as the Randall-Sundrum (RS) graviton at the Tevatron using Run II data collected with the \dzero~detector. Specifically, I searched for the RS graviton when it decays into a $WW$~boson pair. This final state is experimentally attractive because it leads to highly collimated jets when either of the $W$~bosons decays hadronically. When the $W$~boson transverse momentum becomes much greater than its rest mass we observe that the $W$~boson is more likely to be found as a single jet instead of two isolated jets. Using this unique feature of hadronic $W$~boson decays, I extended the current Tevatron limit on RS graviton production into the $WW$~final state by nearly 100~GeV and helped pioneer the use of the jet mass at the Tevatron as a useful quantity when searching for physics beyond the standard model. 

The warped geometry model also provides a myriad of other predictions at the Large Hadron Collider (LHC) and as a Wilson Fellow this will be my primary area of study. At the LHC, the search for hadronically decay $W$~and $Z$~bosons as well as the equally interesting fully hadronic decays of the top quark become more powerful tools due to the increased collision energy and the increased granularity of both the ATLAS and CMS calorimeters. One ``smoking gun" signature of warped extra dimensions are Kaluza-Klein (KK) excitations of the standard model particles. These excitations are the result of the finite extra dimension length similar to the traditional particle in an infinite well from undergraduate quantum mechanics. In particular, the KK excitation(s) of the standard model gluon are predicted to couple more strongly to particles localized closer to our three dimensional world, such as the top quark, compared to other colored objects. This search is experimentally challenging because the strongly interacting KK gluon has a decay width much larger than the detector resolution. Observing the KK gluon as a resonance of two highly boosted top quarks will be one of my principle goals as a Wilson Fellow.

An additional prediction of warped extra dimension models that I will test as a Wilson Fellow is the existence of a massive neutral resonance that is generically called the $Z'$. In these models, the $Z'$~also couples more strongly to heavier particles similar to the KK gluon. If this scenario is true then this particle is most easily seen as a resonance in $e^{+}e^{-}$~and $\mu^{+}\mu^{-}$~final states. Because the muon is two orders of magnitude heavier than the electron we can expect a larger decay width to $\mu^{+}\mu^{-}$ compared to $e^{+}e^{-}$, a scenario not expected in other models containing $Z'$-like particles.

Another program I recently began and would like to continue as a Wilson Fellow is the search for first, second, and third generation leptoquarks. Leptoquarks are Bosonic particles that mediate interactions between leptons and quarks and appear in many grand unified theories. My current interest lies with single production of leptoquarks in association with one lepton, either charged lepton or neutrino depending on the colliding quark flavor. A unique feature of single leptoquark production at the LHC is the large excess of positively charged leptons compared to negatively charged leptons due to the proton-proton initial state. Because these events are result of a gluon-quark collision, they tend to be highly boosted along the direction of the incoming quark. This leads to an experimentally challenging scenario because lepton reconstruction is low in the forward region of the detector. I am currently leading an effort within the ATLAS collaboration to extend the Tevatron sensitivity for leptoquark production and I plan to continue this search as a Wilson Fellow. 

If I am selected as a Wilson Fellow I will bring considerable experience in searches for beyond the standard model physics using a hadron collider. Although not mentioned previously, I am an expert in many advanced multivariate analysis techniques that have proven invaluable at the Tevatron and I believe the same will be true at the LHC. Many of these methods require immense computing resources, which has limited their use in most searches. One program I will implement as a Wilson Fellow is to explore new computing techniques specifically designed to improve the performance of these techniques. I believe a strong collaboration between the computing and experimental communities will dramatically improve the sensitivity of many beyond the standard model searches.

In summary, I ...

\end{document}