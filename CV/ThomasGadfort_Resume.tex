%%%%%%%%%%%%%%%%%%%%%%%%%%%%%%%%%%%%%%%%%%%%%%%%%%%%%%%%%%%%
%
%  Thomas Gadfort's CV
%  Based on R. Van de Water's CV
%  Based on M. Wingate's and David Rainwater's CVs 
%  version of 29 May 2006
%
%%%%%%%%%%%%%%%%%%%%%%%%%%%%%%%%%%%%%%%%%%%%%%%%%%%%%%%%%%%%


\documentclass[12pt]{article}

\usepackage{latexsym,setspace}

%\setstretch{0.5}

\setlength{\oddsidemargin}{0in}
\setlength{\topmargin}{-.5in}
\setlength{\textwidth}{6.5in}
\setlength{\textheight}{9in}
\setlength{\parskip}{0.2in}
%\newcommand{\hr}{\centerline{\hskip 6mm\hrulefill\hskip 6mm}}
\newcommand{\hr}{\centerline{\hskip 30mm\hrulefill\hskip 30mm}}

\newcommand{\dzero}{D\O}


\begin{document}

\pagestyle{empty}

%\vspace{-1cm}\hspace{12cm}{}\emph{\today}

%%%%%%%%%%%%%%%%%%%%%%%%%%%%%%%%%%%%%%%%%%%%%%%%%%%%%%%%%%%%%%%%%%%%%%%%
\vspace{.5cm}

\begin{center}
{\LARGE \textsc{Thomas Gadfort}}

\vspace{-.3cm}\emph{Resume}

\hr

%%%%%%%%%%%%%%%%%%%%%%%%%%%%%%%%%%%%%%%%%%%%%%%%%%%%%%%%%%%%%%%%%%%%%%%%
\noindent
{\sc Contact Information}

\begin{tabular}{lcl}
Fermi National Accelerator Lab		& \hspace{3.47cm} & Telephone: (206) 351-4510 \\
MS122							& \hspace{2.2cm}   & FAX: (630) 840-2347  \\
Batavia, IL 605xx					& \hspace{2.2cm} & E-mail: tgadfort@fnal.gov \\
\end{tabular}

\end{center}


\noindent
{\Large \textbf{Career Objective}}

WRITE SOMETHING HERE

%%%%%%%%%%%%%%%%%%%%%%%%%%%%%%%%%%%%%%%%%%%%%%%%%%%%%%%%%%%%%%%%%%%%%%%%

\pagestyle{myheadings}
\topskip=0.75cm
\headheight=0.75cm
\markright{Thomas Gadfort}


\noindent
{\Large \textbf{Employment History}}
\begin{itemize}
\item{
\underline{Associate Scientist at Fermi National Accelerator Laboratory (2012~-~Present)} \\
Manager for a superconducting magnet critical to muon injection for the $g-2$~experiment. Responsibilities include transportation of the superconducting coils from New York to Illinois and demonstrating the magnet functions at cryogenic temperatures.  \\
Established the reliability of a C++ GEANT-based simulation framework to predict the number of muons delivered to the $g-2$~experiment.} 	\\
\item{
\underline{Goldhaber Fellow at Brookhaven National Laboratory (2009~-~2012)} \\
Managed the ATLAS experimental effort in the search for a new elementary particle called the graviton using data recorded during the 2011 Large Hadron Collider (LHC) run. During my tenure our group published three results in American Physical Society journals Physical Review Letters (PRL) and Physical Review D (PRD) demonstrating the graviton does not exist assuming its mass is less than $1000\times$ the proton mass.}
\item{
\underline{Postdoctoral Reseacher at Columbia University (2007~-~2009)} \\
Led an effort to determine if a new transistor technology based on a silicon-germanium (SiGe) substrate can tolerate high-radiation environments. For this effort, I assembled a highly sensitive current measurement device to profile the newly designed transistors before and after exposure to high levels of neutron radiation. Served as $b$-jet group leader during which we developed a highly efficiency algorithm to identify $b$-jets using the novel multivariate technique known as a boosted decision tree. Certification of this tagger required daily supervision of a Ph. D. student and frequent presentations to the \dzero~experiment management.}
\item{
\underline{Research Assistant at University of Washington (2002~-~2007)}\\
Established the first experimental evidence for a physical process known as ``single top-quark" production using the pioneering technique of matrix element event weighting. The software, based on C++ and FORTRAN, performed multi-dimensional integration based on the high-performing VEGAS routine. \\
On-call expert for this high-level trigger and ethernet-based data acquisition (DAQ) system at \dzero~experiment. Main duties included prompt diagnosis and resolution of data flow stoppage events, daily maintenance of the DELL computer farm, and biannual presentation of DAQ to members of the \dzero~collaboration.}
\end{itemize}


%%%%%%%%%%%%%%%%%%%%%%%%%%%%%%%%%%%%%%%%%%%%%%%%%%%%%%%%%%%%%%%%%%%%%%%%



\noindent
{\Large \textbf{Education}}
\begin{itemize}
\item{
\underline{{\textbf{Ph.~D.}} in high-energy particle physics from the University of Washington (2007)} \\
Title: ``Evidence for Electroweak Top Quark Production in~$p\bar{p}$~Collisions at {\mbox{$\sqrt{s}=1.96$~TeV}}", Eugene Kenneth Miller Award for graduate research.}
\item{
\underline{{\textbf{B.~A.}} in College Scholars (Physics Emphasis) from the University of Tennessee (2001)} \\
Douglas V. Roseberry Award for excellence in Physics, Phi Beta Kappa, Sigma Pi Sigma Physics Honors Society, Outstanding First Year Physics Student}\\
\end{itemize}


%%%%%%%%%%%%%%%%%%%%%%%%%%%%%%%%%%%%%%%%%%%%%%%%%%%%%%%%%%%%%%%%%%%%%%%%
\noindent
{\Large \textbf{Teaching Experience}}
\begin{itemize}
\item{Columbia University (2008): Substitute lecturer for two freshman physics classes. One lecture was on electromagnetic wave properties and the other was on the foundations of quantum mechanics and included a presentation of the Schrodinger equation.}
\item{University of Washington (2001-2002): Laboratory instructor for introductory mechanics.  Tutorial instructor for introductory mechanics, introductory electricity and magnetism, and introductory waves $\&$~optics.}
\item{University of Tennessee (1999): Teaching assistant for computer science course on assembly language and computer organization.}
\end{itemize}


%%%%%%%%%%%%%%%%%%%%%%%%%%%%%%%%%%%%%%%%%%%%%%%%%%%%%%%%%%%%%%%%%%%%%%%%
\noindent
{\Large \textbf{Community Outreach}}
\begin{itemize}
\item{Volunteered at the first annual Science $\&$~Engineering Expo held on the National Mall in Washington, D.C. I worked with several demonstrations designed to illustrate concepts such as Rutherford scattering and neutrino oscillations to children at a middle school to high school level.}
\item{Presented the ATLAS Higgs search at a ``Science Cafe" public lecture held at Borders book store in Setuaket, NY. This presentation was aimed at a general audience with little or no scientific background.}
\end{itemize}


%%%%%%%%%%%%%%%%%%%%%%%%%%%%%%%%%%%%%%%%%%%%%%%%%%%%%%%%%%%%%%%%%%%%%%%%
\noindent
{\Large \textbf{Technology Summary}}

\begin{tabular}{ll}
\underline{Office Applications}			&	Microsoft Office Suite (Word, Excel, PowerPoint) \\
								&	Macintosh OS Suite (Keynote, Pages, Numbers) \\
\underline{Operating Systems}			&	Windows 2000/NT/XP/7, Linux/Unix, Macintosh OS 10.x \\
\underline{Programming Langauges}	&	C, C++, FORTRAN, Visual Basic, Java, Perl \\
								&	BASH and Python scripting \\
\underline{Web Development}			&	HTML, PHP, JavaScript \\
\underline{Database Applications}		&	MySQL, XML, Microsoft Access \\
\underline{Scientific Applications}		&	ROOT, GEANT, GNU Scientific Library (GSL), Mathematica \\
\end{tabular}

\noindent
{\Large \textbf{Personal Information}}

\begin{tabular}{ll}
Birthdate:		&	24 August 1979 \\
Citizenship:	&	Denmark \\
\end{tabular}



\noindent
{\Large \textbf{References}}

\begin{minipage}[t]{.45\textwidth}

{\sc Professor Gordon Watts} \\
University of Washington\\
Department of Physics\\
P.O. Box 351560\\
Seattle, WA 98195-1560\\ 
USA\\[0.1cm]
Telephone: (206) 543-4186\\
FAX:\phantom{iiiiiiiii} (206) 616-9172\\
E-mail: gwatts@phys.washington.edu\\

{\sc Professor Gustaaf Brooijmans} \\
Columbia University\\
Nevis Laboratories \\
136 South Broadway\\
Irvington, NY 10533\\
USA\\[0.1cm]
Telephone:  (914) 591-2804 \\
FAX:\phantom{iiiiiiiii} (914) 591-8120\\
E-mail: gusbroo@nevis.columbia.edu\\

{\sc Professor John Parsons} \\
Columbia University\\
Nevis Laboratories \\
136 South Broadway\\
Irvington, NY 10533\\
USA\\[0.1cm]
Telephone:  (914) 591-2820 \\
FAX:\phantom{iiiiiiiii} (914) 591-8120\\
E-mail: parsons@nevis.columbia.edu\\

%\end{minipage}
%\begin{minipage}[t]{.45\textwidth}

%{\sc Professor John Hobbs} \\
%Physics and Astronomy Department \\
%Stony Brook University \\
%Stony Brook, NY 11794-3800 \\
%USA \\[0.1cm]
%Telephone: (631) 632-8107 \\
%FAX:\phantom{iiiiiiiii} (631) 632-8176\\
%E-mail: John.Hobbs@stonybrook.edu \\

%{\sc Dr. Aurelio Juste} \\
%Institut de F�sica d'Altes Energies (IFAE) \\
%Edifici Cn \\
%Universitat Aut�noma de Barcelona \\
%E-08193 Bellaterra (Barcelona) \\
%Spain \\[0.1cm]
%Telephone:  (+34) 93 581 4249 \\
%FAX:\phantom{iiiiiiiii} (+34) 93 581 1938\\
%E-mail: juste@ifae.es \\

\end{minipage}



\end{document}