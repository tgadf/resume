%%%%%%%%%%%%%%%%%%%%%%%%%%%%%%%%%%%%%%%%%%%%%%%%%%%%%%%%%%%%
%
%  Thomas Gadfort's CV
%  Based on R. Van de Water's CV
%  Based on M. Wingate's and David Rainwater's CVs 
%  version of 29 May 2006
%
%%%%%%%%%%%%%%%%%%%%%%%%%%%%%%%%%%%%%%%%%%%%%%%%%%%%%%%%%%%%


\documentclass[12pt]{article}

\usepackage{latexsym,setspace,color}
\usepackage[normalem]{ulem}

%\setstretch{0.5}

\setlength{\oddsidemargin}{0in}
\setlength{\topmargin}{-.5in}
\setlength{\textwidth}{6.5in}
\setlength{\textheight}{9in}
\setlength{\parskip}{0.2in}
%\newcommand{\hr}{\centerline{\hskip 6mm\hrulefill\hskip 6mm}}
\newcommand{\hr}{\centerline{\hskip 30mm\hrulefill\hskip 30mm}}

\newcommand{\dzero}{D\O}

\begin{document}

\pagestyle{empty}

\vspace{-1cm}\hspace{12cm}{}\emph{\today}

%%%%%%%%%%%%%%%%%%%%%%%%%%%%%%%%%%%%%%%%%%%%%%%%%%%%%%%%%%%%%%%%%%%%%%%%
\vspace{.5cm}

\begin{center}
{\LARGE \textsc{Thomas Gadfort}}

\vspace{-.3cm}\emph{Curriculum Vitae}

\hr

%%%%%%%%%%%%%%%%%%%%%%%%%%%%%%%%%%%%%%%%%%%%%%%%%%%%%%%%%%%%%%%%%%%%%%%%
%\vspace{.3in}
\noindent
{\sc Contact Information}

\begin{tabular}{lcl}
Brookhaven National Lab				& \hspace{3.47cm} & Telephone: (206) 351-4510 \\
Building 510A						& \hspace{2.2cm}   & FAX: (631) 344-5078  \\
Upton, NY 11973					& \hspace{2.2cm} & E-mail: tgadfort@bnl.gov \\
\end{tabular}

\end{center}

%\vspace{.3in}
\noindent
{\Large \textbf{Personal Information}}

\begin{tabular}{ll}
Birthdate:		&	24 August 1979 \\
Citizenship:	&	Denmark \\
\end{tabular}

%%%%%%%%%%%%%%%%%%%%%%%%%%%%%%%%%%%%%%%%%%%%%%%%%%%%%%%%%%%%%%%%%%%%%%%%

\noindent
{\Large \textbf{Education}}
%\underline{\sc Education}

\begin{tabular}{rl}
Fall 2001 - Spring 2007 & Ph.D.\ in Physics \\
	& University of Washington, Seattle \\
	& Dissertation Advisor: Prof. Gordon Watts \\
	& Title: ``Evidence for Electroweak Top Quark Production \\
	& in Proton-Antiproton~Collisions at $\sqrt{s} = 1.96$~TeV" \\
Fall 1997 - May 2001 & B.A.\ in College Scholars with Physics Emphasis, Cum Laude \\	
	& University of Tennessee, Knoxville, TN \\
\end{tabular}


%%%%%%%%%%%%%%%%%%%%%%%%%%%%%%%%%%%%%%%%%%%%%%%%%%%%%%%%%%%%%%%%%%%%%%%%
\noindent
{\Large \textbf{Awards and Honors}}
%\underline{\sc Awards and Honors}

\begin{tabular}{rl}
March 2009			& BNL Goldhaber Distinguished Fellowship \\
May 2004				& Eugene Kenneth Miller Award for excellence in graduate research \\
May 2000				& Douglas V. Roseberry Award for excellence in Physics \\
December 1999	 	& Phi Beta Kappa  \\
May 1999				& Sigma Pi Sigma Physics Honors Society \\
May 1998				& Outstanding First Year Physics Student \\
\end{tabular}

\noindent
{\Large \textbf{Professional Research Experience}}

\begin{tabular}{ll}
October 2009~$-$~Present			&	Goldhaber Distinguished Fellow \\
								&	Brookhaven National Laboratory \\
Spring 2007~$-$~October 2009		&	Postdoctoral Researcher \\
								&	Columbia University \\
\end{tabular}

%%%%%%%%%%%%%%%%%%%%%%%%%%%%%%%%%%%%%%%%%%%%%%%%%%%%%%%%%%%%%%%%%%%%%%%%

\newpage
\pagestyle{myheadings}
\topskip=0.75cm
\headheight=0.75cm
\markright{Thomas Gadfort}

\noindent
\begin{singlespace}

\noindent
{\Large \textbf{Selected Physics Analyses}}

\begin{itemize}

\item{\underline{Search for scalar and vector leptoquark production with the ATLAS detector}\\
I am currently participating in a search for $1^{\rm{st}}$ generation leptoquarks using $\sqrt{s}=7$~TeV LHC data collected with the ATLAS detector. Leptoquarks, or an equivalent particle that couples leptons and quarks, appear in many extensions to the standard model including grand unified theories and R-parity violating supersymmetry. Our group will publish this work based on the first $35$~pb$^{-1}$ of ATLAS data and we expect to dramatically improve the current mass limits on this particle beyond those set by the \dzero~and CDF collaborations.}

\item{\underline{Search for diboson resonances and other new phenomena with the \dzero~detector}\\In collaboration with other \dzero~members, I have been searching for the production of two types of beyond the standard model particles produced in $p\bar{p}$~collisions using up to $5.4$~fb$^{-1}$~of \dzero~Run II~data. The first search is for a massive electrically charged resonance decaying to $W$~and $Z$~boson pairs using events with at least one charged lepton resulting from either the $W$ or $Z$~boson decay. Using the sequential standard model (SSM) $W'$~boson as a baseline prediction, we excluded a large fraction of the model parameter space defined in terms of the $W'$~mass and the $W'WZ$~coupling strength. This result was published in PRL [Phys.\ Rev.\  Lett. {\bf 104}, 061801 (2010)]. I also contributed to the search for electroweak production of vector-like quarks at the Tevatron. This analysis sets the first lower limits on the vector-like quark production assuming both charged current ($Q\rightarrow Wq$) and neutral current ($Q\rightarrow Zq$) decays, and has been submitted for publication in PRL [arXiv:1010.1466].}

\item{\underline{Discovery of single top quark production with the \dzero~detector}\\I applied an analysis technique known as the matrix element method to obtain the first $3\sigma$~evidence for single top quark production in $p\bar{p}$~collisions using $1$~fb$^{-1}$~of \dzero~Run II~data. The matrix element method takes advantage of all angular correlations in both the signal production and decay through a probability based on the product of the leading-order single top quark matrix element and the expected energy and momentum resolution of the final state particles. This work resulted in two publications PRL [Phys. Rev. Lett. {\bf 98}, 181802 (2007)] and PRD [Phys. Rev. D {\bf 78}, 012005 (2008)]. In collaboration with two other \dzero~members, I improved this method in a subsequent analysis using 2.3 fb$^{-1}$~of Run II data. This analysis measured the single top quark production cross section at the $5\sigma$~confidence level and was published in PRL [Phys.\ Rev.\ Lett. {\bf 103}, 092001 (2009)].}
%
%\item{\underline{Search for the Higgs Boson with the \dzero~detector.}\\I applied the matrix element analysis technique to both the low mass ($M_{H}<135$~GeV) and high mass ($M_{H}>135$~GeV) Higgs search at \dzero. This method is particularly useful in the low mass search, where the Higgs is expected to decay primarily into two $b$-jets, by naturally including not only the dijet mass, but the dijet mass resolution into a final discriminating variable. This method was shown to improve the sensitivity to a low mass Higgs by up to $10\%$~over the previous result.}

\end{itemize}

\clearpage
\noindent
{\Large \textbf{Computing Experience}}
\begin{itemize}
\item{\underline{Operating Systems}\\
I have more than ten years of experience with {\sc unix/linux} and several Windows OS releases including XP and Vista. In addition, I have six years of experience with the MAC OS X platform.}
\item{\underline{Programming Languages}\\ I am highly experienced with {\sc C}, {\sc C++},  and {\sc Python};  I also have a working knowledge of {\sc Fortran}, {\sc PHP}, {\sc Make} and {\sc Java}.}
\item{\underline{HEP Software}\\ I have more than seven years of experience with the {\sc ROOT} analysis framework. This includes several software packages integrated with {\sc ROOT} such as {\sc TMVA} and {\sc Qt}. I have working knowledge of the {\sc Pythia}, {\sc MadGraph}, and {\sc CalcHEP} Monte Carlo generators, as well as of the {\sc GSL} and {\sc CERNLIB} software libraries.}
\item{\underline{Distributed Computing}\\ I have seven years of experience with the \dzero~analysis software. This includes running analysis jobs on the local desktop computing cluster known as Clued0 and on the central analysis backend (CAB) computing farm. I also have two years of experience with GRID computing using the {\sc ATHENA} software framework of the ATLAS experiment. In addition, I have one year of experience with a local CONDOR computing cluster at Brookhaven National Lab.}
\item{\underline{Data Handling} \\
I was a 24/7 on-call expert for the level 3 trigger and data acquisition system at \dzero. The level 3 trigger and DAQ system is designed to collect data from the each detector sub-system upon a level 2 trigger accept and channel this data to one of many computers on which the level 3 trigger software will reduce the flow-rate from $700$~Hz to $50$~Hz. The data from each sub-system is collected on a single board computer (SBC) and transferred to the level 3 computer farm via optical fiber and a $16$~GHz CISCO 6049C ethernet switch. My primary duties as an expert were to quickly diagnose and resolve problems to prevent data flow stoppage, to distribute centrally compiled software to worker nodes, and to give annual tutorials on the system for new data taking shifters.}
\end{itemize}
%
%\item{\underline{Liquid Argon Front End Board Development for the ATLAS sLHC Upgrade.}\\ The sLHC is expected to increase the instantaneous luminosity of $pp$~collisions at the LHC by a factor of 10, resulting in an increase in the radiation exposure for the the calorimeter electronics. New transistor technologies that will form the basis of analog-to-digital converters (ADCs) are currently being tested to determine if they will be viable in this high radiation environment. For this effort, I assembled a highly sensitive current measurement device which can profile the newly designed transistors before and after exposure to radiation. I also began work on the next generation front end electronic readout board which will test ADCs of various material types to determine if they meet the stringent requirements necessary to collect data at the sLHC. }


%
%{\sc \large{\dzero~Editorial Review Board}}
%\begin{itemize}
%\item{Served on the editorial review board for the $t\bar{t}\rightarrow$``all-jets" (6~jets) and the $t\bar{t}\rightarrow\tau_{h}\rm{+jets}$ analyses (2008-2011). This committee reviews these two analyses and determines if they are advanced enough to be presented publicly. Sample duties include: thorough review of internal analysis documents and suggesting cross-checks to determine stability of the result,  including a review of the systemic uncertainties. This board approved a publication in Physical Review D. based on the measurement of the $t\bar{t}$~cross section in the all-hadronic channel [Phys.\ Rev.\  D {\bf 82}, 032002 (2010)].}
%\end{itemize}
\clearpage

\noindent
{\Large \textbf{Management and Teaching Experience}}

\begin{itemize}
\item{\underline{Co-convener of the $b$-jet identification group at \dzero~(2007-2009)}\\My main duties as convener included supervising Ph.D. students, coordinating with \dzero~management and physics groups, and developing new algorithms to increase $b$-jet efficiency and light-jet rejection. Notable achievements during my tenure include: (1) Determining $b$-jet, $c$-jet, and light-jet efficiencies for the entire Run IIb dataset corresponding to $2.8~\rm{fb}^{-1}$ of \dzero~data. (2) Supervising work on a new algorithm designed to distinguish $b$-jets from $c$-jets using a multivariate analysis technique known as a boosted decision tree.  This technique resulted in an additional $15\%$ rejection of charm-jets.  (3) Helping to increase understanding of jets produced from parton-parton interactions versus jets produced from low $p_{T}$ $p\bar{p}$ interactions. This work is expected to increase signal acceptance by $5\%$ in the low-mass Higgs search.}

\item{\underline{Undergraduate and graduate student supervision}\\ I am currently collaborating with a fifth-year graduate student at Stony Brook University on the search for first generation scalar and vector leptoquark production at the LHC using the ATLAS detector. Previously, I collaborated with a Columbia graduate student whose thesis topic was the search for vector-like quark production at the Tevatron. This student is currently a postdoc with Michigan State University. I also supervised a Columbia graduate student attempting to find neutral long lived particles decaying to $b\bar{b}$~pairs using $4~\rm{fb}^{-1}$~of Run IIa and RunIIb~\dzero~data. The student graduated in 2009 and is currently a postdoc with Indiana University.}
%\item{Daily supervision of an undergraduate student during the summers of 2008 and 2009.}

\item{\underline{Undergraduate physics and computer science teaching} \\ In 2008 at Columbia University, I was a substitute lecturer for two freshman physics classes. One lecture was on electromagnetic wave properties and the other was on the foundations of quantum mechanics and included a presentation of the Schr\"odinger equation. During 2001 and 2002 at the University of Washington I was a laboratory instructor for introductory mechanics, as well as a tutorial instructor for introductory mechanics, introductory electricity and magnetism, and introductory waves $\&$~optics. In 1998 at the University of Tennessee, I was a teaching assistant for a computer science course on assembly language and computer organization.}

\end{itemize}

%\clearpage
%{\sc \large{Trigger Development}}
%\begin{itemize}
%\item{Developed an extension to the trigger used to select single top quark events (2005). This new tool allowed the trigger to select an isolated muon from a leptonic $W$ decay when it is produced in association with a high $p_{T}$ jet. The result of this work was an increased rejection of heavy flavor (i.e. $B\rightarrow \mu$+X) background allowing the trigger to record events event at very high instantaneous luminosities.}
%\item{Developed an alternative strategy to select top quark events using a muon+jets trigger as well as a suite of single muon triggers (2005-2006). This new technique was the first successful ``OR" of triggers used to select top quark events and resulted in a $10\%$ increase in potential single top quark candidate events.}
%\item{Lead author on software package to simulate trigger response on Monte Carlo events based on trigger efficiencies measured in data (2006). This package is used by nearly all analyses with a leptonically-decaying $W$ or $Z$-boson in the final state.}
%\end{itemize}


%%%%%%%%%%%%%%%%%%%%%%%%%%%%%%%%%%%%%%%%%%%%%%%%%%%%%%%%%%%%%%%%%%%%%%%%
%\noindent


%%%%%%%%%%%%%%%%%%%%%%%%%%%%%%%%%%%%%%%%%%%%%%%%%%%%%%%%%%%%%%%%%%%%%%%%
\noindent
{\Large \textbf{Outreach}}
\begin{itemize}
\item{\underline{Volunteer at the first annual Science $\&$~Engineering Expo in Washington, D.C.} \\ In October 2010 I worked with members of the Fermilab Science Education Office and other physicists on several demonstrations designed to illustrate concepts such as Rutherford scattering and neutrino oscillations to children at a middle school to high school level.}
\end{itemize}



%\clearpage

\end{singlespace}
%
%\noindent
%{\Large \textbf{Other Research Experience}}

%\textbf{Spring 2003 - Summer 2003, KATRIN Experiment}

%\qquad Calibrated silicon pin diodes used to detect $18$~keV electrons emitted from a Tritium beta decay. This work required designing and constructing preamplifiers for signal enhancement as well as assembling a data acquisition system.

%\textbf{Summer 2002 - Spring 2003, SNO Experiment}

%\qquad Measured electric discharge rates of readout cables used in the neutral current detectors (NCDs). Also assisted in leak testing for Radium and Thorium contamination on the remotely operated vehicle (ROV) which was used to install the NCDs in the SNO detector.

%\textbf{Summer 2000, KamLAND Experiment}

%\qquad Worked alongside a team of physicists retrofitting 20" photo multiplier tubes (PMTs) to work in the KamLAND experiment. Also spent one month at the KamLAND experiment installing the PMTs and cleaning the detector.


%%%%%%%%%%%%%%%%%%%%%%%%%%%%%%%%%%%%%%%%%%%%%%%%%%%%%%%%%%%%%%%%%%%%%%%%
\clearpage
\noindent
{\Large \textbf{Conference Talks}}
%\underline{\sc Conference Talks}

\begin{tabular}{rl}
23 August 2010	& 	``Jet substructure measurements wth ATLAS" \\
				&	US ATLAS Hadronic Final State Forum: \\
				&	Joint Theory/Experiment Open Session \\
23 June 2010		&	``Boosted Objects at the Tevatron"\\
				&	Boost 2010, Oxford University	\\
8 June 2010		&	``LHC Experimental Update" \\
				&	RHIC \& AGS Annual Users' Meeting \\
15 March 2009		&	``Searches for low mass SM Higgs at the Tevatron" \\
				&	{\sc{XLIV}}th Recontres de Moriond QCD, La Thuile, Italy \\
7 Nov 2008		&	``$W'$~and $Z'$~Discovery Potential with the ATLAS Detector"\\
				&	Brookhaven Forum 2008, Upton, NY\\
15 May 2007		&	``Measuring Single Top Quark Production at \dzero~With\\
				&	 The Matrix Element Method"\\
				&	{\sc{XLII}}nd Recontres de Moriond EW, La Thuile, Italy \\
2 June 2006		&	``Search for Electroweak Top Quark Production at \dzero" \\
				&	New Perspectives, FNAL \\
13 May 2005		&	``Search for Electroweak Top Quark Production" \\
				&	Northwest APS Meeting, Victoria, BC \\
22 May 2004		&	``Search for Single Top Production at \dzero" \\
				&	Northwest APS Meeting, Moscow, ID \\
2 May 2004		&	``Search for Single Top Quark Production in the Muon+Jets Channel" \\
				&	APS April Meeting, Denver, CO \\
\end{tabular}


%%%%%%%%%%%%%%%%%%%%%%%%%%%%%%%%%%%%%%%%%%%%%%%%%%%%%%%%%%%%%%%%%%%%%%%%
\noindent
{\Large \textbf{Seminars}}

\begin{tabular}{rl}
27 January 2010	&	``Search for $W'\rightarrow WZ$ at \dzero"\\
				&	BNL, Upton, NY	\\
3 October 2006		&	``Search for Electroweak Top Quark Production at \dzero"\\
				&	FNAL, Batavia, IL	\\
21 September 2006	&	``Search for Electroweak Top Quark Production at \dzero"\\
				&	Boston University, Boston, MA	\\
13 September 2006	&	``Search for Electroweak Top Quark Production at \dzero"\\
				&	Columbia University, New York, NY	\\
27 January 2005	&	``The Search for Electroweak Top Quark Production at \dzero" \\
				&	University of Washington, Seattle, WA \\
\end{tabular}




%%%%%%%%%%%%%%%%%%%%%%%%%%%%%%%%%%%%%%%%%%%%%%%%%%%%%%%%%%%%%%%%%%%%%%%%
\noindent
{\Large \textbf{Workshops and Schools Attended}}
%\underline{\sc Workshops and Schools Attended}

\begin{tabular}{rl}
18 June -- 1 July 2006	&	European School of High Energy Physics \\
					&	Aronsburg, Sweden \\
15 June -- 27 June 2003	&	Nuclear Physics Summer School \\
					&	University of Tennessee, Knoxville \\
\end{tabular}

%%%%%%%%%%%%%%%%%%%%%%%%%%%%%%%%%%%%%%%%%%%%%%%%%%%%%%%%%%%%%%%%%%%%%%%%
\newpage


%%%%%%%%%%%%%%%%%%%%%%%%%%%%%%%%%%%%%%%%%%%%%%%%%%%%%%%%%%%%%%%%%%%%%%%%
\noindent
{\Large \textbf{Selected Publications}}

\vspace{-2mm}
\begin{enumerate}

\item{ V.~M.~Abazov {\it et al.} ``Search for single vector-like quarks in ppbar collisions at sqrt(s) = 1.96 TeV," Submitted to Phys.\ Rev.\  Lett. (2010) [arXiv:1010.1466 [hep-ex]].
}

\item{  V.~M.~Abazov {\it et al.}  [D0 Collaboration],
  ``Search for a Resonance Decaying into $WZ$ Boson Pairs in $p\bar{p}$~Collisions,''
  Phys.\ Rev.\  Lett. {\bf 104}, 061801 (2010)
  [arXiv:0912.0715 [hep-ex]].
  %%CITATION = PHRVA,D78,012005;%%
}

\item{  V.~M.~Abazov {\it et al.}  [D0 Collaboration], ``Observation of Single Top-Quark Production,"
  Phys.\ Rev.\ Lett. {\bf 103}, 092001 (2009)
  [arXiv:0903.0850 [hep-ex]].
}

\item{  V.~M.~Abazov {\it et al.}  [D0 Collaboration],
  ``Search for resonant pair production of neutral long-lived particles decaying to bbbar in ppbar collisions at sqrt(s)=1.96 TeV,''
  Phys.\ Rev.\ Lett. {\bf 103}, 071801 (2009)
  [arXiv:0912.0715 [hep-ex]].
  %%CITATION = PHRVA,D78,012005;%%
}

%\cite{Abazov:2008kt}
\item{  V.~M.~Abazov {\it et al.}  [D0 Collaboration],
  ``Evidence for production of single top quarks,''
  Phys.\ Rev.\  D {\bf 78}, 012005 (2008)
  [arXiv:0803.0739 [hep-ex]].
  %%CITATION = PHRVA,D78,012005;%%
}

%\cite{Abazov:2006gd}
\item{  V.~M.~Abazov {\it et al.}  [D0 Collaboration],
  ``Evidence for production of single top quarks and first direct measurement of $|V_{tb}|$,''
  Phys.\ Rev.\ Lett.\  {\bf 98}, 181802 (2007)
  [arXiv:hep-ex/0612052].
}

%\cite{Abazov:2006uq}
\item{  V.~M.~Abazov {\it et al.}  [D0 Collaboration],
  ``Multivariate searches for single top quark production with the D0
  detector,''
  Phys.\ Rev.\  D {\bf 75}, 092007 (2007)
  [arXiv:hep-ex/0604020].
  %%CITATION = PHRVA,D75,092007;%%
}
  
\item{  V.~M.~Abazov {\it et al.}  [D0 Collaboration],
  ``Search for single top quark production in p anti-p collisions at  s**(1/2)
  = 1.96-TeV,''
  Phys.\ Lett.\  B {\bf 622}, 265 (2005)
  [arXiv:hep-ex/0505063].
  %%CITATION = PHLTA,B622,265;%%
}

\end{enumerate}

\noindent
{\Large \textbf{Selected D\O~Internal Notes}}
%\underline{\sc D$\O$ Internal Notes}
\vspace{-2mm}
\begin{enumerate}

\item{``Search for Vector-like Quark Production in the Lepton+Jets Final State Using 5.4 fb$^{-1}$~of Run II Data" \\
Gustaaf Brooijmans, Seth Caughron, Thomas Gadfort, Lidija Zivkovic. (\dzero~Note Number 6103).} 

\item{``Search for the first generation leptoquark pair production in the electron + neutrino + jets final state in 5.4 fb$^{-1}$ of Run II Data" \\
Xinlu Huang, Thomas Gadfort, Gustaaf Brooijmans, Lidija Zivkovic. (\dzero~Note Number 6023).}

\item{``Search for $W'\rightarrow WZ$~production in the lepton+jets and 2 dilepton+jets final states using 5.4 fb$^{-1}$~of Run II Data" \\
Andrew Askew, Gustaaf Brooijmans, Seth Caughron, Thomas Gadfort, Ketino Kaadze, Yurii, Maravin, Lidija Zivkovic. (\dzero~Note Number 6022).}

\item{``Search for a resonance in $WZ\rightarrow 3\ell\nu$~production in 4.1 fb$^{-1}$~of Run II data"\\
Andrew Askew, Thomas Gadfort, Ketino Kaadze, Yurii Maravin.  (\dzero~Note Number 5969). }

\item{``Search for a Neutral Long-Lived Particle Decaying to b-Jets with the \dzero~Detector" \\
Chad Johnson, Gustaaf Brooijmans, Thomas Gadfort and Andy Haas. (\dzero~Note Number 5881).}

\item{``Performance of the DO NN b-tagging Tool on 3 fb-1 Run IIb Data"\\
Tim Scanlon, Thomas Gadfort. (\dzero~Note Number 5803).}

\item{``Search for the Higgs boson in $H\rightarrow WW\rightarrow \mu\mu$ decays at \dzero~using Neural Networks with $3.0~\rm{fb}^{-1}$ of Data" \\
T. Gadfort, A. Haas, D. Johnston, B. Penning. (\dzero~Note Number 5787).}

\item{``Combination of the \dzero~$H\rightarrow WW\rightarrow \ell\ell\nu\nu$~limits with 3.0 fb$^{-1}$~at \dzero~in Run II"\\ 
Ralf Bernhard, Harald Fox, Thomas Gadfort, Andy Haas, Dale Johnston, Bjoern Penning. (\dzero~Note Number 5757).}

\item{``p20 ICR Electron Identification"\\
James Kraus, Thomas Gadfort, Oleksiy Atramentov. (\dzero~Note Number 5691).}

\item{``Search for $H\rightarrow WW\rightarrow \ell\ell$~using 1.2 fb$^{-1}$~of Run IIb Data" \\
Ralf Bernhard, Thomas Gadfort, Andy Haas, Bjoern Penning. (\dzero~Note Number 5624).}

\item{``Search for $H\rightarrow WW\rightarrow \mu\mu$~using 1.2 fb$^{-1}$~of Run IIb Data" \\
Thomas Gadfort, Andy Haas. (\dzero~Note Number 5623).}

\item{``Performance of the Dzero NN b-tagging Tool on p20 Data" \\
T. Gadfort, A. Haas, D. Johnston, D. Lincoln, T. Scanlon, S. Schlobohm. (\dzero~Note Number 5554)}

\item{``Search for WH Production Using The Matrix Element Analysis Technique in $900~\rm{pb}^{-1}$ of Data Collected with the \dzero~Detector" \\
T. Gadfort, D. Johnston, A. Juste, A. Magerkurth, J. Qian, J. Strandberg, C. Xu. (\dzero~Note Number 5365)}

\item{``Search for Single Top Quark Production using the Matrix Element Analysis Technique in $1~\rm{fb}^{-1}$~of Data" \\
E. Aguilo et. al. (\dzero~Note Number 5287)}

\item{``Muon Identification Certification for p17 data'' \\
P. Calfayan, T. Gadfort, G. Hesketh, V. Lesne, M. Owen, R. Stroehmer, B. Tuchming. (\dzero~Note Number 5157)} 

\item{``ORing single muon triggers and one muon+jets trigger" \\
Thomas Gadfort. (\dzero~Note Number 4899)}

\end{enumerate}

%%%%%%%%%%%%%%%%%%%%%%%%%%%%%%%%%%%%%%%%%%%%%%%%%%%%%%%%%%%%%%%%%%%%%%%%
\newpage
\noindent
{\Large \textbf{References}}

\begin{itemize}
\item{Professor Gordon Watts (Thesis Advisor)\\
University of Washington\\
Department of Physics\\
P.O. Box 351560\\
Seattle, WA 98195-1560\\ 
USA\\[0.1cm]
Telephone: (206) 543-4186\\
FAX:\phantom{iiiiiiiii} (206) 616-9172\\
E-mail: gwatts@phys.washington.edu\\
}

\item{Professor Gustaaf Brooijmans\\
Columbia University\\
Nevis Laboratories \\
136 South Broadway\\
Irvington, NY 10533\\
USA\\[0.1cm]
Telephone:  (914) 591-2804 \\
FAX:\phantom{iiiiiiiii} (914) 591-8120\\
E-mail: gusbroo@nevis.columbia.edu\\
}

\item{Professor John Parsons\\
Columbia University\\
Nevis Laboratories \\
136 South Broadway\\
Irvington, NY 10533\\
USA\\[0.1cm]
Telephone:  (914) 591-2820 \\
FAX:\phantom{iiiiiiiii} (914) 591-8120\\
E-mail: parsons@nevis.columbia.edu\\
}

\item{Professor John Hobbs \\
Physics and Astronomy Department \\
Stony Brook University \\
Stony Brook, NY 11794-3800 \\
USA \\[0.1cm]
Telephone: (631) 632-8107 \\
FAX:\phantom{iiiiiiiii} (631) 632-8176\\
E-mail: John.Hobbs@stonybrook.edu \\
}
\end{itemize}

\end{document}