%%%%%%%%%%%%%%%%%%%%%%%%%%%%%%%%%%%%%%%%%%%%%%%%%%%%%%%%%%%%
%
%  Thomas Gadfort's CV
%  Based on R. Van de Water's CV
%  Based on M. Wingate's and David Rainwater's CVs 
%  version of 29 May 2006
%
%%%%%%%%%%%%%%%%%%%%%%%%%%%%%%%%%%%%%%%%%%%%%%%%%%%%%%%%%%%%


\documentclass[12pt]{article}

\usepackage{latexsym,setspace}

%\setstretch{0.5}

\setlength{\oddsidemargin}{0in}
\setlength{\topmargin}{-.5in}
\setlength{\textwidth}{6.5in}
\setlength{\textheight}{9in}
\setlength{\parskip}{0.2in}
%\newcommand{\hr}{\centerline{\hskip 6mm\hrulefill\hskip 6mm}}
\newcommand{\hr}{\centerline{\hskip 30mm\hrulefill\hskip 30mm}}

\newcommand{\dzero}{D\O}
\newcommand{\prd}{Phys.\ Rev.\ D}
\newcommand{\prl}{Phys.\ Rev.\ Lett.}
\newcommand{\plb}{Phys.\ Lett.\ B}

\begin{document}

\pagestyle{empty}

\vspace{-1cm}\hspace{12cm}{}\emph{\today}

%%%%%%%%%%%%%%%%%%%%%%%%%%%%%%%%%%%%%%%%%%%%%%%%%%%%%%%%%%%%%%%%%%%%%%%%
\vspace{.5cm}

\begin{center}
{\LARGE \textsc{Thomas Gadfort}}

\vspace{-.3cm}\emph{Curriculum Vitae}

\hr

%%%%%%%%%%%%%%%%%%%%%%%%%%%%%%%%%%%%%%%%%%%%%%%%%%%%%%%%%%%%%%%%%%%%%%%%
\vspace{.3in}
\noindent
{\sc Contact Information}

\begin{tabular}{lcl}
Fermi National Accelerator Lab		& \hspace{3.47cm} & Telephone: (206) 351-4510 \\
MS122							& \hspace{2.2cm}   & FAX: (630) 840-2347  \\
Batavia, IL 60510					& \hspace{2.2cm} & E-mail: tgadfort@fnal.gov \\
\end{tabular}

\end{center}

\vspace{.3in}
\noindent
{\Large \textbf{Personal Information}}

\begin{tabular}{ll}
Birthdate:		&	24 August 1979 \\
Citizenship:	&	Denmark \\
\end{tabular}

%%%%%%%%%%%%%%%%%%%%%%%%%%%%%%%%%%%%%%%%%%%%%%%%%%%%%%%%%%%%%%%%%%%%%%%%

\noindent
{\Large \textbf{Education}}
%\underline{\sc Education}

\begin{tabular}{rl}
Fall 2001 - Spring 2007 & Ph.D.\ in Physics \\
	& University of Washington, Seattle \\
	& Dissertation Advisor: Prof. Gordon Watts \\
	& Title: ``Evidence for Electroweak Top Quark Production \\
	& in Proton-Antiproton~Collisions at $\sqrt{s} = 1.96$~TeV" \\
Fall 1997 - May 2001 & B.A.\ in College Scholars with Physics Emphasis, Cum Laude \\	
	& University of Tennessee, Knoxville, TN \\
\end{tabular}


%%%%%%%%%%%%%%%%%%%%%%%%%%%%%%%%%%%%%%%%%%%%%%%%%%%%%%%%%%%%%%%%%%%%%%%%
\noindent
{\Large \textbf{Awards and Honors}}
%\underline{\sc Awards and Honors}

\begin{tabular}{rl}
March 2009			& BNL Goldhaber Distinguished Fellowship \\
May 2004				& Eugene Kenneth Miller Award for excellence in graduate research \\
May 2000				& Douglas V. Roseberry Award for excellence in Physics \\
December 1999	 	& Phi Beta Kappa  \\
May 1999				& Sigma Pi Sigma Physics Honors Society \\
May 1998				& Outstanding First Year Physics Student \\
\end{tabular}

%%%%%%%%%%%%%%%%%%%%%%%%%%%%%%%%%%%%%%%%%%%%%%%%%%%%%%%%%%%%%%%%%%%%%%%%

\newpage
\pagestyle{myheadings}
\topskip=0.75cm
\headheight=0.75cm
\markright{Thomas Gadfort}

\noindent
{\Large \textbf{Research Experience}}

\begin{tabular}{ll}
Fall 2012~$-$~Present				&	Associate Scientist \\
								&	$(g-2)_{\mu}$~experiment \\
								&	Fermi National Accelerator Laboratory \\
Fall 2009~$-$~Fall 2012				&	Goldhaber Distinguished Fellow \\
								&	ATLAS experiment \\
								&	Brookhaven National Laboratory \\
Spring 2007~$-$~Fall 2009			&	Postdoctoral Researcher \\
								&	ATLAS and \dzero~experiments \\
								&	Columbia University \\
Fall 2003~$-$~Spring 2007			&	Research Assistant \\
								&	\dzero~experiment \\
								&	University of Washington \\
%Fall 2002~$-$~Fall 2003				&	Research Assistant \\
%								&	KATRIN experiment \\
%								&	University of Washington \\
%Fall 2000~$-$~Spring 2001			&	Senior Thesis \\
%								&	FOCUS experiment \\
%								&	University of Tennessee \\
\end{tabular}



%\begin{singlespace}

{\sc \large{Physics Programs}}

\begin{itemize}

\item{\underline{\bf{Measurement of the muon anomalous magnetic moment (2012-Present)}} \\
I am currently developing a GEANT4-based simulation framework for the E989 \mbox{``g-2"}~experiment, which will measure the muon anomalous magnetic moment, $a_{\mu}\equiv (g-2)/2$, to 0.14 ppm precision. The primary goal of the simulation is to accurately predict the number of muons available for for analysis within the E989 muon storage ring. This requires modeling a mix of time-dependent electric and magnetic fields as well as predicting the background due to muon losses early in the measurement time window. A early version of this work was incorporated into the E989 conceptual design review (CDR).}

\item{\underline{\bf{Search for leptoquark production (2010-2012)}} \\
I completed a search for first and second generation scalar leptoquarks using $1$~fb$^{-1}$~of $\sqrt{s} = 7$~TeV LHC data collected with the ATLAS detector. Leptoquarks, or an equivalent particle that couples leptons and quarks, appear in many extensions to the standard model including grand unified theories and R-parity violating supersymmetry. This result will be submitted for publication soon. The previous year I published a search for leptoquarks using $35$~pb$^{-1}$. I found no evidence for leptoquark production and excluded the possibility of first and second generation scalar leptoquarks below approximately 400 GeV. This work was published in Phys. Rev. D {\bf{83}} 112006 (2011) [arXiv:1104.448].}

\item{\underline{\bf{Search for new physics with $W$~and $Z$~bosons (2008-2012)}}\\
I searched for the production of three specific beyond the standard model particles produced in $p\bar{p}$~collisions using 5.4 fb$^{-1}$~of \dzero~Run II data. The first search is for a massive electrically charged resonance decaying to $W$~and $Z$~boson pairs using events with at least one charged lepton resulting from either the $W$~or $Z$~boson decay. No excess of data was observed in this search. Using the extended gauge model (EGM) $W'$~boson as a baseline prediction, I excluded a large fraction of the model parameter space defined by the $W'$~mass and the $W'WZ$~coupling strength. Also included in this result is a limit on low scale Technicolor, where the $\rho_{T}$~is equivalent to the EGM $W'$. Both of these results can be found in the PRL publication [Phys. Rev. Lett. {\bf{104}}, 061801 (2010)]. Another search that was published by PRL [Phys. Rev. Lett. {\bf{106}}, 081801 (2011)] is the electroweak production of vector-like quarks. This analysis sets the first lower limits on the vector-like quark production assuming both charged current ($Q \rightarrow Wq$) and neutral current ($Q \rightarrow Zq$) decays. The final search, whose result is near publication, is the production of a Randall-Sundrum graviton decaying to a $WW$~boson pair. This search uses events with a semi-leptonic $WW$~decay and is the first Tevatron Run II analysis to use events with a single massive jet, which result from highly boosted hadronic $W$~and $Z$~decays.}

\item{\underline{\bf{Single top quark production cross section measurement (2003-2009)}} \\
I obtained the first evidence for single-top quark production and then the first measurement of the single-top quark production cross section using an analysis technique known as the matrix element method. The matrix element method takes advantage of all angular correlations in both the signal production and decay through a probability based on the product of the leading-order single top quark matrix element and the expected energy and momentum resolution of the final state particles. This work resulted in two publications in PRL [Phys. Rev. Lett. {\bf{98}}, 181802 (2007)] and PRD [Phys. Rev. D {\bf{78}}, 012005 (2008)]. I improved this method in a subsequent analysis using $2.3$~fb$^{-1}$~of Run II data. This analysis measured the single top quark production cross section at the $5\sigma$~confidence level and was published in PRL [Phys. Rev. Lett. {\bf{103}}, 092001 (2009)].}
\end{itemize}

{\sc \large{Leadership Roles and Student Supervision}}
\begin{itemize}
\item{\underline{\bf{Paper editor and co-convener of diboson resonance search group~(2011-2012)}} \\
I lead the ATLAS effort to search for resonant $WW$,~$WZ$,~or $ZZ$~production using data recorded during the 2011 LHC run at $\sqrt{s} = 7$~TeV. Together with my co-convener we brought two analyses to a publication level in the past six months. These analyses search for 1) Technicolor in $\rho_{T} \rightarrow WZ \rightarrow \ell\nu\ell\ell$~final states and 2) Randall-Sundrum graviton in $G^{*}_{KK} \rightarrow ZZ \rightarrow \ell\ell\ell\ell$~and $\ell\ell jj$~final states. In the coming year we expect these searches will be published along with two additional analyses: Search for the radion (a spin-0 particle found in models with extra dimensions) in $\ell\ell\ell\ell$~final states and a search for a $Z'$~or $G^{*}_{KK}$~in $WW \rightarrow \ell\nu\ell\nu$~and $\ell\nu jj$~events.}

\item{\underline{\bf{Co-convener of the $b$-jet identification group at \dzero~(2007-2009)}}\\
As a $b$-jet convener, I supervised PHD students, frequently interacted with \dzero~management and physics analysis groups, and developed new algorithms to increase $b$-jet efficiency and light-jet rejection. Notable achievements during my tenure include, but are not limited to: (1) Determined $b$-jet, $c$-jet, and light-jet efficiencies for the entire Run IIb dataset corresponding to $2.8~\rm{fb}^{-1}$ of \dzero~data. (2) Supervised work on a new algorithm designed to distinguish $b$-jets from $c$-jets using a multivariate analysis technique known as a boosted decision tree. This technique resulted in an additional $15\%$ rejection of charm-jets. (3) Helped to increase understanding of jets produced from parton-parton interactions versus jets produced from low $p_{T}$ $p\bar{p}$ interactions. This work increased signal acceptance by $5\%$ in the low-mass Higgs search.}

\item{I recently completed a year-long research project with a student from the Illinois Math and Sciences Academy (IMSA), who designed and constructed a pair of electro-magnets with the goal of producing a region of space with a negligible magnetic field (2012-2013).}
\item{I collaborated with two Columbia University graduate students on the search resonant diboson production at the LHC using the ATLAS detector (2011-2013). Both students are postdocs in the field of medical physics.}
\item{I collaborated with a graduate student from Stony Brook University on the search for first generation scalar leptoquark production at the LHC using the ATLAS detector (2010-2011).  This student is currently a postdoc working on the ATLAS experiment.}
\item{I collaborated with a Columbia graduate student whose thesis topic was the search for vector-like quark production at the Tevatron (2009-2010). This student is now a consultant in Boston, MA.}
\end{itemize}

%\clearpage

{\sc \large{Hardware Experience and Service Tasks}}
\begin{itemize}

\item{\underline{\bf{$(g-2)_{\mu}$~silicon sensor beam monitor (2012-2013)}}\\
I am currently developing a muon beam monitor for the E989 experiment using Tevatron Run II~silicon hybrid sensors. The beam monitor will be located in a region of the E989 storage ring that is too small for traditional wire plane beam monitors, thus silicon is the preferred technology choice. The silicon sensor data is extracted using commodity hardware previously developed for the CMS~experiment.}

\item{\underline{\bf{ATLAS event reconstruction software (2011)}}\\
I worked with a team charged with daily maintenance of the ATLAS �job transforms�, which are python scripts designed to facilitate simulation and reconstruction of both Monte Carlo and LHC data events in a transparent manner. Duties include responding to �bug reports� as needed and improving functionality to the end users.}

\item{\underline{\bf{Liquid Argon Front End Board Development (2007-2009)}}\\
The sLHC is expected to increase the instantaneous luminosity of $pp$~collisions at the LHC by a factor of 10, resulting in an increase in the radiation exposure for the the calorimeter electronics. While based at Columbia University, I led an effort to characterize a new transistor technology based on a silicon-germanium (SiGe) substrate that has been shown to tolerate high radiation environments. For this effort, I assembled a highly sensitive current measurement device to profile the newly designed transistors before and after exposure to high levels of radiation.}

\item{\underline{\bf{$t\bar{t} \rightarrow$~``all-jets" (6 jets) Editorial Board Member (2008-2010)}}\\
This committee reviews the $t\bar{t} \rightarrow$~``all-jets"~and the $t\bar{t} \rightarrow \tau_{h}$+jets analyses and determines if they are advanced enough to be presented publicly. My editorial board duties included thoroughly reviewing internal analysis documents and suggesting cross-checks to determine the stability and reliability of the result, including a review of the systemic uncertainties. This board approved a publication in Physical Review D of the measurement of the $t\bar{t}$~cross section in the all-hadronic channel [Phys. Rev. D {\bf{82}}, 032002 (2010)].}

\item{\underline{\bf{Level 3 / Data Acquisition System (2003-2007)}}\\
The level 3/DAQ system is designed to collect data from the each detector sub-system upon a level 2 trigger accept and channel this data to one of many computers on which the level 3 trigger software will reduce the flow-rate from 700 Hz to 50 Hz. The data from each sub-system is collected on a single board computer (SBC) and transferred to the level 3 computer farm via optical fiber and a 16 GHz CISCO 6049C ethernet switch. I was an on-call expert for this system at \dzero. My main duties as an expert were to quickly diagnose and resolve problems to prevent data flow stoppage, provide daily maintenance of the level 3 computer farm, and give bi-yearly tutorials on the system for new data taking shifters.}
\end{itemize}

%%%%%%%%%%%%%%%%%%%%%%%%%%%%%%%%%%%%%%%%%%%%%%%%%%%%%%%%%%%%%%%%%%%%%%%%
\clearpage
\noindent
{\Large \textbf{Conference Talks}}
%\underline{\sc Conference Talks}

\begin{tabular}{rl}
20 January 2014	&	``Taking the Muon for a Spin" \\
				&	ASPEN Physics Institute, Aspen, CO \\
16 April 2012		&	``Searches for Physics Beyond the Standard Model \\
				&	using the ATLAS Detector" \\
				&	University of Michigan, Ann Arbor, MI	\\
19 April 2011		&	``Use of the jet mass in the reconstruction of hadronic $W$~and \\
				&	$Z$~boson decays in the \dzero~experiment" \\
				&	Jet reconstruction and spectroscopy at hadron colliders\\
				&	INFN, Pisa, Italy \\
23 August 2010	& 	``Jet substructure measurements wth ATLAS" \\
				&	US ATLAS Hadronic Final State Forum: \\
				&	Joint Theory/Experiment Open Session \\
23 June 2010		&	``Boosted Objects at the Tevatron"\\
				&	Boost 2010, Oxford University	\\
8 June 2010		&	``LHC Experimental Update" \\
				&	RHIC and AGS Annual Users' Meeting \\
15 March 2009		&	``Searches for low mass SM Higgs at the Tevatron" \\
				&	{\sc{XLIV}}th Recontres de Moriond QCD, La Thuile, Italy \\
7 Nov 2008		&	``$W'$~and $Z'$~Discovery Potential with the ATLAS Detector"\\
				&	Brookhaven Forum 2008, Upton, NY\\
15 May 2007		&	``Measuring Single Top Quark Production at \dzero~With\\
				&	 The Matrix Element Method"\\
				&	{\sc{XLII}}nd Recontres de Moriond EW, La Thuile, Italy \\
2 June 2006		&	``Search for Electroweak Top Quark Production at \dzero" \\
				&	New Perspectives, FNAL \\
13 May 2005		&	``Search for Electroweak Top Quark Production" \\
				&	Northwest APS Meeting, Victoria, BC \\
22 May 2004		&	``Search for Single Top Production at \dzero" \\
				&	Northwest APS Meeting, Moscow, ID \\
2 May 2004		&	``Search for Single Top Quark Production in the Muon+Jets Channel" \\
				&	APS April Meeting, Denver, CO \\
\end{tabular}


%%%%%%%%%%%%%%%%%%%%%%%%%%%%%%%%%%%%%%%%%%%%%%%%%%%%%%%%%%%%%%%%%%%%%%%%
\noindent
{\Large \textbf{Seminars}}

\begin{tabular}{rl}
11 April 2012		&	``Searches for the Higgs Boson and Beyond the \\
				&	Standard Model Physics with the ATLAS Detector" \\
				&	Fordham University, Bronx, NY 	\\
20 January 2012	&	``Searches for the Higgs Boson and New Physics \\
				&	with the ATLAS Detector" \\
				&	FNAL, Batavia, IL	\\
13 May 2011		&	``Search for new physics with leptons and jets" \\
				&	BNL, Upton, NY	\\
13 May 2011		&	``Search for new physics with leptons and jets" \\
				&	BNL, Upton, NY	\\
27 January 2010	&	``Search for $W'\rightarrow WZ$ at \dzero"\\
				&	BNL, Upton, NY	\\
3 October 2006		&	``Search for Electroweak Top Quark Production at \dzero"\\
				&	FNAL, Batavia, IL	\\
21 September 2006	&	``Search for Electroweak Top Quark Production at \dzero"\\
				&	Boston University, Boston, MA	\\
13 September 2006	&	``Search for Electroweak Top Quark Production at \dzero"\\
				&	Columbia University, New York, NY	\\
27 January 2005	&	``The Search for Electroweak Top Quark Production at \dzero" \\
				&	University of Washington, Seattle, WA \\
\end{tabular}




%%%%%%%%%%%%%%%%%%%%%%%%%%%%%%%%%%%%%%%%%%%%%%%%%%%%%%%%%%%%%%%%%%%%%%%%
\noindent
{\Large \textbf{Workshops and Schools Attended}}
%\underline{\sc Workshops and Schools Attended}

\begin{tabular}{rl}
25 - 27 April 2013				&	Intensity Frontier Workshop \\
							&	FNAL \\
11 - 13 October 2012			&	Community Planning Meeting \\
							&	FNAL \\
30 November -- 2 December 2011	&	Fundamental Physics at the Intensity Frontier \\
							&	Rockville, Maryland \\
18 June -- 1 July 2006			&	European School of High Energy Physics \\
							&	Aronsburg, Sweden \\
15 June -- 27 June 2003			&	Nuclear Physics Summer School \\
							&	University of Tennessee, Knoxville \\
\end{tabular}

%%%%%%%%%%%%%%%%%%%%%%%%%%%%%%%%%%%%%%%%%%%%%%%%%%%%%%%%%%%%%%%%%%%%%%%%
%\newpage


%%%%%%%%%%%%%%%%%%%%%%%%%%%%%%%%%%%%%%%%%%%%%%%%%%%%%%%%%%%%%%%%%%%%%%%%
\noindent
{\Large \textbf{Selected Publications}}

\vspace{-2mm}
\begin{enumerate}

\item{G. Aad {\it et al.} [ATLAS Collaboration], ``Search for resonant diboson production in the $WW/WZ\rightarrow\ell\nu jj$~decay channels with the ATLAS detector at~$\sqrt{s}=7$~TeV," \prd~{\bf 87}  112006 (2013).}

\item{G. Aad {\it et al.} [ATLAS Collaboration], ``Observation of a new particle in the search for the Standard Model Higgs boson with the ATLAS detector at the LHC", \plb~{\bf 716} 1-29 (2012).}

\item{ G.~Aad {\it et al.} [ATLAS Collaboration],   ``Search for pair production of first or second generation leptoquarks in proton-proton collisions at sqrt(s)=7 TeV using the ATLAS detector at the LHC,� \prd~{\bf{83}}, 112006 (2011).}

\item{ V.~M.~Abazov {\it et al.} [\dzero~Collaboration], 
  ``Search for Resonant $WW$~and $WZ$~Production in $p\bar{p}$~Collisions at $\sqrt{s} = 1.96 $~TeV,� \prl~{\bf{107}}, 011801 (2011).}

\item{ V.~M.~Abazov {\it et al.} [\dzero~Collaboration], 
  ``Search for single vector-like quarks in ppbar collisions at sqrt(s) = 1.96 TeV," \prl~{\bf{106}}, 081801 (2011).}

\item{ V.~M.~Abazov {\it et al.} [\dzero~Collaboration], 
  ``Search for a Resonance Decaying into $WZ$~Boson Pairs in $p\bar{p}$~Collisions,� \prl~{\bf{104}}, 061801 (2010).}

\item{  V.~M.~Abazov {\it et al.}  [\dzero~Collaboration], ``Observation of Single Top-Quark Production,"
  \prl~{\bf 103}, 092001 (2009).}
%
%\item{  V.~M.~Abazov {\it et al.}  [\dzero~Collaboration],
%  ``Evidence for production of single top quarks,''
%  Phys.\ Rev.\  D {\bf 78}, 012005 (2008)
%  [arXiv:0803.0739 [hep-ex]].}
%  
%\item{  V.~M.~Abazov {\it et al.}  [\dzero~Collaboration],
%  ``Evidence for production of single top quarks and first direct measurement of $|V_{tb}|$,''
%  Phys.\ Rev.\ Lett.\  {\bf 98}, 181802 (2007)
%  [arXiv:hep-ex/0612052].}

\end{enumerate}
%
%\noindent
%{\Large \textbf{Selected ATLAS~Internal Notes}}
%\vspace{-2mm}
%\begin{enumerate}
%\item{``Search for first generation scalar leptoquarks in $pp$~collisions at $\sqrt{s} = 7$~TeV", ATL-COM-PHYS-2011-1363 (11-30-2011).}
%\item{``Searches for second generation leptoquarks in dimuon plus jets and muon, missing ET plus jets �nal states using the ATLAS detector", ATL-COM-PHYS-2011-1376 (11-30-2011).}
%\item{``Search for resonant $ZZ$~production at $\sqrt{s} = 7$~TeV using the ATLAS detector", ATL-COM-PHYS-2011-1456 (11-29-2011).}
%\item{``Search for resonant $WZ$~production in the $WZ \rightarrow \ell\nu\ell\ell$~channel using the ATLAS detector", ATL-COM-PHYS-2011-1188 (11-24-2011).}
%\item{``Search for New Physics in Events with Four Charged Leptons", ATLAS-COM-CONF-2011-143 (11-03-2011).}
%\item{``Search for resonant $WW$~production in fully leptonic boson decays using the ATLAS detector ", ATL-COM-PHYS-2011-1362 (11-02-2011).}
%\item{``Search for resonant $WZ$~and $ZZ$~production in the $WZ/ZZ \rightarrow \ell\ell qq$~channel using the ATLAS detector", ATL-COM-PHYS-2011-931 (11-07-2011).}
%\item{ ``Search for pair production of first or second generation leptoquarks in proton-proton collisions at $\sqrt{s} = 7$~TeV using the ATLAS detector at the LHC� (ATLAS-EXOT-2010-08 (22-04-2011).}
%\item{ ``Evaluation of Two SiGe HBT Technologies for the ATLAS sLHC Upgrade� \\
%	Presented at Topical Workshop on Electronics for Particle Physics by Miguel Ullan, Naxos, Greece, 15 - 19 Sep 2008, pp.111-115.}
%\end{enumerate}
%
%\noindent
%{\Large \textbf{Selected D\O~Internal Notes}}
%%\underline{\sc D$\O$ Internal Notes}
%\vspace{-2mm}
%\begin{enumerate}
%
%\item{``Search for Vector-like Quark Production in the Lepton+Jets Final State Using 5.4 fb$^{-1}$~of Run II Data" \\
%Gustaaf Brooijmans, Seth Caughron, Thomas Gadfort, Lidija Zivkovic. (\dzero~Note Number 6103).} 
%
%\item{``Search for $W'\rightarrow WZ$~production in the lepton+jets and 2 dilepton+jets final states using 5.4 fb$^{-1}$~of Run II Data" \\
%Andrew Askew, Gustaaf Brooijmans, Seth Caughron, Thomas Gadfort, Ketino Kaadze, Yurii, Maravin, Lidija Zivkovic. (\dzero~Note Number 6022).}
%
%\item{``Performance of the DO NN b-tagging Tool on 3 fb-1 Run IIb Data"\\
%Tim Scanlon, Thomas Gadfort. (\dzero~Note Number 5803).}
%
%\item{``Search for the Higgs boson in $H\rightarrow WW\rightarrow \mu\mu$ decays at \dzero~using Neural Networks with $3.0~\rm{fb}^{-1}$ of Data" \\
%T. Gadfort, A. Haas, D. Johnston, B. Penning. (\dzero~Note Number 5787).}
%
%\item{``Combination of the \dzero~$H\rightarrow WW\rightarrow \ell\ell\nu\nu$~limits with 3.0 fb$^{-1}$~at \dzero~in Run II"\\ 
%Ralf Bernhard, Harald Fox, Thomas Gadfort, Andy Haas, Dale Johnston, Bjoern Penning. (\dzero~Note Number 5757).}
%
%\item{``Performance of the Dzero NN b-tagging Tool on p20 Data" \\
%T. Gadfort, A. Haas, D. Johnston, D. Lincoln, T. Scanlon, S. Schlobohm. (\dzero~Note Number 5554)}
%
%\item{``Search for Single Top Quark Production using the Matrix Element Analysis Technique in $1~\rm{fb}^{-1}$~of Data" \\
%E. Aguilo et. al. (\dzero~Note Number 5287)}
%
%\item{``Muon Identification Certification for p17 data'' \\
%P. Calfayan, T. Gadfort, G. Hesketh, V. Lesne, M. Owen, R. Stroehmer, B. Tuchming. (\dzero~Note Number 5157)} 
%
%\end{enumerate}

%%%%%%%%%%%%%%%%%%%%%%%%%%%%%%%%%%%%%%%%%%%%%%%%%%%%%%%%%%%%%%%%%%%%%%%%
\noindent
{\Large \textbf{Teaching Experience}}
%\underline{\sc Teaching Experience}

\textbf{2008, Columbia University}

\qquad Substitute lecturer for two freshman physics classes. One lecture was on electromagnetic wave properties and the other was on the foundations of quantum mechanics and included a presentation of the Schrodinger equation.

\textbf{2001 - 2002, University of Washington}

\qquad Laboratory instructor for introductory mechanics.  Tutorial instructor for introductory mechanics, introductory electricity and magnetism, and introductory waves $\&$~optics.
%
%\textbf{1998, University of Tennessee}
%
%\qquad Teaching assistant for computer science course on assembly language and computer organization.
%


%%%%%%%%%%%%%%%%%%%%%%%%%%%%%%%%%%%%%%%%%%%%%%%%%%%%%%%%%%%%%%%%%%%%%%%%
\noindent
{\Large \textbf{Outreach}}
%\underline{\sc Teaching Experience}
\begin{itemize}
\item{Volunteered at the first annual Science $\&$~Engineering Expo held on the National Mall in Washington, D.C. I worked with several demonstrations designed to illustrate concepts such as Rutherford scattering and neutrino oscillations to children at a middle school to high school level.}

\item{Presented the ATLAS Higgs search at a ``Science Cafe" public lecture held at Borders book store in Setuaket, NY. This presentation was aimed at a general audience with little or no scientific background.}
\end{itemize}


%%%%%%%%%%%%%%%%%%%%%%%%%%%%%%%%%%%%%%%%%%%%%%%%%%%%%%%%%%%%%%%%%%%%%%%%
\newpage
\noindent
{\Large \textbf{References}}

\begin{minipage}[t]{.45\textwidth}

{\sc Professor Gordon Watts} \\
University of Washington\\
Department of Physics\\
P.O. Box 351560\\
Seattle, WA 98195-1560\\ 
USA\\[0.1cm]
Telephone: (206) 543-4186\\
FAX:\phantom{iiiiiiiii} (206) 616-9172\\
E-mail: gwatts@phys.washington.edu\\

{\sc Professor Gustaaf Brooijmans} \\
Columbia University\\
Nevis Laboratories \\
136 South Broadway\\
Irvington, NY 10533\\
USA\\[0.1cm]
Telephone:  (914) 591-2804 \\
FAX:\phantom{iiiiiiiii} (914) 591-8120\\
E-mail: gusbroo@nevis.columbia.edu\\

{\sc Professor John Parsons} \\
Columbia University\\
Nevis Laboratories \\
136 South Broadway\\
Irvington, NY 10533\\
USA\\[0.1cm]
Telephone:  (914) 591-2820 \\
FAX:\phantom{iiiiiiiii} (914) 591-8120\\
E-mail: parsons@nevis.columbia.edu\\

%\end{minipage}
%\begin{minipage}[t]{.45\textwidth}

%{\sc Professor John Hobbs} \\
%Physics and Astronomy Department \\
%Stony Brook University \\
%Stony Brook, NY 11794-3800 \\
%USA \\[0.1cm]
%Telephone: (631) 632-8107 \\
%FAX:\phantom{iiiiiiiii} (631) 632-8176\\
%E-mail: John.Hobbs@stonybrook.edu \\

%{\sc Dr. Aurelio Juste} \\
%Institut de F�sica d'Altes Energies (IFAE) \\
%Edifici Cn \\
%Universitat Aut�noma de Barcelona \\
%E-08193 Bellaterra (Barcelona) \\
%Spain \\[0.1cm]
%Telephone:  (+34) 93 581 4249 \\
%FAX:\phantom{iiiiiiiii} (+34) 93 581 1938\\
%E-mail: juste@ifae.es \\

\end{minipage}


\end{document}